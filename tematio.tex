\documentclass[letterpaper]{report}
\usepackage{pst-node}
\usepackage{tikz-cd} 
\usepackage{amsmath}
\usepackage{float}
\usepackage{amsfonts}
\usepackage{amssymb}
\usepackage[spanish,activeacute]{babel}
\usepackage{amscd}
\usepackage{color}
\usepackage{transparent}
\usepackage{afterpage}
\usepackage{array}

\renewcommand{\labelenumii}{\theenumii}
\renewcommand{\theenumii}{\theenumi.\arabic{enumii}.}

\begin{document}
\noindent\Huge{\textbf{Temario}}
\normalsize
\vspace{0.5cm}
\begin{enumerate}
\item\textbf{Introducci\'on: Demostraci'on del teorema de Chow por medio de los resultados de E. Bishop como inspiraci\'on para un puente entre geometr\'ia algebraica y an\'alisis complejo.}
\item\textbf{Resultados de Bishop}
    \begin{enumerate}
        \item Variedades anal\'iticas y algebraicas
        \item El crecimiento del volumen como medida para determinar si un conjunto es algebraico
        \item Teorema secuencial de variedades anal\'iticas de Bishop, m\'etrica de Hausdorff y una nueva demostraci\'on del teorema de Montel
        \item Similitudes entre teoremas cl\'asicos de variable compleja y los resutados de Bishop
    \end{enumerate}
\item\textbf{Aplicaci\'on a foliaciones holomorfas en variedades K\"ahlerianas}
    \begin{enumerate}
        \item Variedades K\"ahler, subvariedades y sus vol\'umenes
        \item La funci\'on volumen de las hojas y el resultado de Edwards Millet y Sullivan para variedades K\"ahler
        \item Estabilidad de las hojas en variedades K\"ahler, la conjetura de Beauville y avances en su demostraci\'on
        \item Una posible demostraci\'on del teorema de Beauville para foliaciones cuya funci\'on de volumen en las hojas sea localmente acotada
    \end{enumerate}
\item\textbf{Comportamiento local de foliaciones por curvas, hojas completas y sus l\'imites}
    \begin{enumerate}
        \item La corona como modelo ara estudiar foliaciones holomorfas uniformizadas por el disco
        \item Foliaciones por discos en conjuntos pseudoconvexos, l\'imites y especulaciones
    \end{enumerate}
\item\textbf{Variedades de Fano-Poisson}
    \begin{enumerate}
        \item Variedades de Poisson, el teorema de Weinstein y su foliaci\'on por hojas simpl\'ecticas
        \item Variedades de Fano-Poisson vistas desde la perpectiva de la geometr\'ia K\"ahleriana
        \item La conjeura de Bondal e hip\'otesis sobre los vol\'umenes de sus conjuntos singulares
    \end{enumerate}
\end{enumerate}
\end{document}
