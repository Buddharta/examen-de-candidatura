\documentclass{article}
\usepackage{pst-node}
\usepackage{tikz-cd} 
\usepackage{amsmath}
\usepackage{float}
\usepackage{amsfonts}
\usepackage{amssymb}
\usepackage[spanish,activeacute]{babel}
\usepackage{amscd}
\usepackage{color}
\usepackage{transparent}
\graphicspath{{./figs/}}
\usepackage{makeidx}
\usepackage{afterpage}
\usepackage{array}


\newtheorem{teorema}{Teorema}[section]
\newtheorem{prop}[teorema]{Proposici\'on}
\newtheorem{cor}[teorema]{Corolario}
\newtheorem{lema}[teorema]{Lema}
\newtheorem{def.}{Definici\'on}[section]
\newtheorem{afir}{Afirmaci\'on}
\newtheorem{conjetura}{Conjetura}

\renewcommand{\figurename}{Figura}
%\renewcommand{\chaptername}{\Large  \sc Cap\'{\i}tulo}
\renewcommand{\indexname}{\'{I}ndice anal\'{\i}tico}
\newcommand{\zah}{\ensuremath{ \mathbb Z }}
\newcommand{\nat}{\ensuremath{ \mathbb N }}
%\renewcommand{\bibname}{Bibliograf\'{\i}a}
\newcommand{\dem}{{\sc Demostraci\'on. }}
\newcommand{\bg}{\ensuremath{\overline \Gamma}}
\newcommand{\ga}{\ensuremath{\Gamma}}
\newcommand{\fb}{\ensuremath{\overline f}}
\newcommand{\la}{\ensuremath{\lambda}}
\newcommand{\La}{\ensuremath{\Lambda}}
\newcommand{\bt}{\ensuremath{\overline T}}
\newcommand{\li}{\ensuremath{\mathbb{L}}}
\newcommand{\ord}{\ensuremath{\mathbb{O}}}
\newcommand{\bs}{\ensuremath{\mathbb{S}^1}}
\newcommand{\co}{\ensuremath{\mathbb C }}
\newcommand{\con}{\ensuremath{\mathbb{C}^n}}
\newcommand{\cp}{\ensuremath{\mathbb{CP}}}
\newcommand{\rp}{\ensuremath{\mathbb{RP}}}
\newcommand{\re}{\ensuremath{\mathbb R }}
\newcommand{\hc}{\ensuremath{\widehat{\mathbb C} }}
\newcommand{\pslz}{\ensuremath{PSL(2,\mathbb Z) }}
\newcommand{\pslr}{\ensuremath{PSL(2,\mathbb R) }}
\newcommand{\pslc}{\ensuremath{PSL(2,\mathbb C) }}
\newcommand{\qed}{\ensuremath{\hspace*{0em plus 1fill}\blacksquare}}
\newcommand{\hd}{\ensuremath{\mathbb H^2}}

%%\makeindex
%%\renewcommand*{\contentsname}{Temario}
\begin{document}
\begin{titlepage}
\begin{center}

\LARGE\textbf{ UNIVERSIDAD NACIONAL AUT'ONOMA DE M\'EXICO}
\vspace*{0.3cm}

\small PROGRAMA DE MAESTR'IA Y DOCTORADO EN CIENCIAS MATEM'ATICAS Y DE LA ESPECIALIZACI'ON EN ESTAD'ISTICA APLICADA
\vspace*{2cm}

\large\textbf{PRESENTACI\'ON DE PROYECTO DE INVESTIGACI'ON DOCTORAL PARA EL EXAMEN DE CANDIDATURA}
\vspace*{2cm}

\small\textbf{PRESENTA:}
\vspace*{0.2cm}

\small CARLOS EDUARDO MART'INEZ AGUILAR
\vspace*{0.2cm}

\small TUTOR: DR. SANTIAGO ALBERTO VERJOVSKY SOL'A
\vspace*{0.2cm}

\small \textbf{INSTITUTO DE MATEM'ATICAS UNAM}
\vspace*{2.5cm}
\end{center}

\end{titlepage}

\tableofcontents
\clearpage

\section{Introducci\'on}
\noindent Presento a continuaci\'on el temario y avances de las investigaciones realizadas durante los primeros cinco semestres 
de mi precandidatura a estudiante doctoral como parte del proceso de examen de candidatura a doctorante.
Durante el periodo indicado, tanto mi director de tesis (Dr. Santiago Alberto Verjovsky Sol\'a) como yo nos dedicamos a estudiar 
foliaciones holomorfas $\mathfrak{F}$ cuyas hojas cumplan con ciertas propiedades geom\'etricas que permiten garantizar cierta regularidad en 
la estructura del espacio de hojas de la foliaci\'on. Nos referimos a regularidad en el sentido de que normalmente el 
espacio de hojas es topol\'ogicamente complejo (no Hausdorff), por lo que es necesario restringir el tipo de din\'amicas en las hojas 
de la foliaci\'on para garantizar cierta regularidad topol\'ogica en el espacio de hojas. Un ejemplo de estas propiedades es el caso de foliaciones cuyas hojas tengan 
vol\'umenes acotados en variedades de K\"ahler $(M,h)$. 
La base hist\'orica de estudiar foliaciones con volumen acotado proviene de trabajos previos hechos por matem\'aticos como
Epstein, Edwards, Millet, Reeb, Sullivan, Haeflinger, J.C. Alexander y el mismo Dr.Verjovsky, entre otros, v\'ease \cite{Epstein1}, \cite{Epstein2}, \cite{EMS}, \cite{E-V}, \cite{V-A}. 
Sin embargo, la investigaci\'on que hemos propuesto hasta ahora se encuentra apoyada por el trabajo previo de J.C. Alexander y el 
Dr.Verjovsky \cite{V-A} el cual a su vez se basa en trabajos de Erret Bishop \cite{Bishop} sobre extensiones y l\'imites de Hausdorff 
de sucesiones de variedades anal\'iticas y sigue el esp\'iritu de la presentaci\'on de los resultados de Bishop que se presentan en el 
libro ``Volumes, Limits and Extensions of Analytic Varieties'' de Gabriel Stolzenberg \cite{Stolzenberg}.

M\'as precisamente, hemos estado investigando maneras de extender el rango de las posibles aplicaciones de estos resultados a
distintas \'areas del an\'alisis complejo y la geometr\'ia compleja, con \'enfasis en la teor\'ia de foliaciones holomorfas. En el trabajo
ya realizado hemos encontrado conexiones entre teoremas conocidos del an\'alisis complejo y la geometr\'ia compleja e incluso hemos
escrito un pequeño art\'iculo de divulgaci\'on con estos hallazgos, el cual por el momento se encuentra en una etapa de correcciones
y posibles expansiones. Uno de los primeros hallazgos de estas conexiones es, por ejemplo, una pueba novedosa del teorema 
de Chow \cite{Chow}
\section{Resultados de Bishop}
\subsection{Variedades anal\'iticas y algebraicas}
\begin{teorema}[Chow]\label{Chow}
        Todo subespacio cerrado anal\'itico del espacio proyectivo complejo $\cp^{n}$ es algebraico.
\end{teorema}
\noindent Demostado a partir de los resultados de Bishop sobre variedades anal\'iticas de dimensi\'on pura $k$. Aqu\'i se define como \emph{espacio 
anal\'itico} o \emph{variedad anal\'itica} a un espacio topol\'ogico Hausdorff $X$ paracompacto con una estructura anillada $\mathcal{H}_X$, es decir, una 
gavilla de anillos en $X$ que es localmente isomorfa a un subconjunto anal\'itico de $\con$. M\'as precisamente, un conjunto es anal\'itico 
si para cada $x\in X$ existe una vecindad abierta $V$ y un homeomorfismo $\varphi_V:V\rightarrow Z,$ donde $U\subset\con$ es un abierto y $Z=Z(f_1,\dots,f_k)(U)$ 
representa al subconjunto anal\'itico de los ceros de las funciones holomorfas $f_i:U\rightarrow\co$ y el pullback $\varphi^{*}:\mathcal{H}(U)/\mathcal{I}(Z)\rightarrow \mathcal{H}_X(V)$ 
es un isomorfismo de anillos entre $\mathcal{H}_X(V)$ y el anillo cociente de
\[
        \mathcal{H}(U)=\{f:U\rightarrow\co\,|\,f \text{ es holomorfa}\}\hspace{0.2cm}\text{con}\hspace{0.2cm}\mathcal{I}(Z)=\{f\in\mathcal{H}(U)\,|\,f|_{Z}=0\}.
\]
As\'i, un subespacio anal\'itico $Y$ de $X$ es un espacio anillado $(Y,\mathcal{H}_Y)$ con un encaje cerrado $\iota:Y\rightarrow X$ tal que localmente el anillo
$\mathcal{H}_Y$ es isomorfo al anillo cociente de $\mathcal{H}$ con el ideal de funciones nulas en la imagen de $Y$, donde el isomorfismo se da por medio del 
pullback $\iota^{*}$. Similarmente, la definici\'on de conjunto algebraico se obtiene reemplazando el anillo de funciones anal\'iticas $\mathcal{H}(U)$ con el anillo de 
polinomios en $\con$ denotado por $\co[z_1,\dots,z_n]$.
\subsection{El crecimiento del volumen como medida para determinar si un conjunto es algebraico}
\noindent Siguiendo la presentacion del libro de Stolzenberg (v\'ease \cite{Stolzenberg}) el resultado que liga 
los resultados de Bishop con el teorema de Chow es un resultado que estipula lo siguiente:
\begin{teorema}[Bishop]\label{Bishop1}
        Sea $X\subset\con$ una subvariedad anal\'itica de dimensi\'on compleja pura $k$, si para toda $R\in\re^+$
        $$\textrm{Vol}_{2k}(X\cap B_R(0))\leq CR^{2k},$$
        donde $B_R(0)$ denota a la bola centrada en el origen de radio $R$, $C\in\re^{+}$ es una constante positiva
        y $\textrm{Vol}_{2k}$ es el volumen de dimensi\'on $2k$. Entonces $X$ es algebraica.
\end{teorema}
\noindent La demostraci\'on que dimos en nuestro art\'iculo del teorena de Chow es de un car\'acter muy distinto a la demostraci\'on 
famosa de Serre (v\'ease \cite{GAGA}) y nos habla de una conexi\'on interesante entre an\'alisis y geometr\'ia algebraica.

\subsection{El teorema secuencial de Bishop, la m\'etrica de Haussdorff y una demostraci\'on del teorema de Montel}
\noindent Las conexiones entre el an\'alisis y la geometr\'ia compleja nos han guiado a explorar m\'as sobre las posibles conexiones entre estas dos ramas,
adem\'as de otros resultados con el an\'alisis complejo. Un ejemplo es el siguiente resultado cl\'asico que demostramos
a partir de los resultados de Bishop. 
\begin{teorema}[Montel]\label{Montel}
        Sea $B\subset\con$ la bola unitaria, si $\mathcal{H}(\overline{B})$ es el \'algebra de Banach de funciones holomorfas que son continuas 
        en la frontera de $B$, entonces toda familia de funciones localmente acotadas $F\subset\mathcal{H}(\overline{B})$ es una familia \textit{normal}.
\end{teorema}
Como recordatorio hay que mencionar que una familia de funciones $F$ es \textit{normal} si y s\'olo si su cerradura es secuencialmente compacta. Si vemos este 
teorema desde la perspectiva de los teoremas de Bishop, entonces podemos considerar que el siguiente teorema es una generalizaci\'on en
un contexto m\'as general (v\'ease \cite[p. 30]{Stolzenberg}).

\begin{teorema}[Bishop]\label{Bishop2}
        Sea $\{ V_n \}_{n\in\nat}$ una sucesi\'on de subvariedades anal\'iticas de dimensi\'on compleja pura $k$ de una regi\'on 
        $\Omega\subset\con$ con volumen uniformemente acotado por una constante $C\in\re^{+}$.
        Si $V_n\rightarrow V$ converge a un cerrado $V\subset\con$ en el sentido de Hausdorff, entonces $V$ es una 
        variedad anal\'itica. 
\end{teorema}
Esto debido a que Dr. Verjovsky y yo demostramos el resultado de Montel por medio del teorema \ref{Bishop2} pensando a las gr\'aficas de las
funciones como variedades anal\'iticas
\[
        \Gamma_{f}=\{(z,w)\in B\times\co\,|\,w=f(z)\}=Z(w-f(z)),
\]
\noindent donde para una funci\'on holomorfa $g:\Omega\rightarrow\co$ denotamos a su conjunto nulo como $Z(g)=\{z\in\Omega\,|\,g(z)=0\}$. Ahora aqu\'i 
pensamos a $\Gamma_{f}$ como variedades anal\'iticas de dimensi\'on pura $n$ en un abierto de $\co^{n+1}$, as\'i demostramos 
que para toda sucesi\'on $\{f_n\}_{n\in\nat}\subset F$ de funciones en una familia localmente acotada, tiene una subsucesi\'on 
cuyas gr\'aficas convergen en el sentido de la m\'etrica de Hausdorff. Adem\'as, demostramos que su conjunto l\'imite 
es a su vez la gr\'afica de una funci\'on holomorfa en $B$.

Convergencia de una sucesi\'on de conjuntos $S_n\rightarrow S$ \emph{en el sentido de Hausdorff} para conjuntos cerrados de un espacio m\'etrico $(X,d)$
sucede cuando $S_n\cap K\rightarrow S\cap K$ en la m\'etrica de Hausdorff $d_H$ para todo $K\subset X$ compacto, donde \'esta se define como
\[
        d_H(K_1,K_2):= \max_{x\in K_1}\left\{d(x,K_2)\right\}+\max_{y\in K_2}\left\{d(y,K_1)\right\}. 
\]

\subsection{Similitudes entre teoremas cl\'asicos de an\'alisis complejo y los resultados de Bishop}

\noindent Se puede apreciar desde la perspectiva que he mostrado hasta ahora que los resultados de Bishop se pueden pensar 
como generalizaciones de resutados cl\'asicos del an\'alisis complejo, pero en el contexto de la teor\'ia de variedades anal\'iticas.

\begin{table}[hpt]
        \caption{Tabla con similitudes entre teoremas cl\'asicos de variable compleja y los teoremas de Bihop.}\label{Tab}
        \centering
        \begin{tabular}{|m{5.5cm}|m{5.5cm}|} \hline
                        \begin{center} \vspace*{0.2cm} 
                                \underline{\textbf{An\'alisis Complejo}} 
                        \end{center} & 
                        \begin{center} \vspace*{0.2cm}
                                \underline{\textbf{Conjuntos anal\'iticos en $\con$}}
                        \end{center} \\
                \hline
                \begin{center} 
                        Teorema de Liouville. 
                \end{center} & 
                \begin{center}
                        Teorema de Bishop (Teorema \ref{Bishop1}).
                \end{center}\\ 
                        \hline Si $\vert f(z)\vert\leq C\,R^k$ en el conjunto $\{\vert z\vert\leq R\}$ 
                        para todo $R\in\re^{+}$con $f$ entera y $k\in\zah^{+}$, entonces 
                        $f$ es un polinomio. 
                        &
                        \vspace{0.1cm}
                        Si $\textrm{Vol}_{2k}(X\cap B(R,0))\leq CR^{2k}$ para todo
                        $R\in\re^{+}$, donde $X$ es una subvariedad anal\'itica en $\con$, entonces $X$ es algebraica.\\ 
                        \hline
                        \vspace{0.1cm}
                        \begin{center} 
                        Teorema de extensi\'on de Riemann. 
                \end{center} 
                        & 
                \begin{center} 
                        Generalizaci\'on del teorema de Remmert-Stein de Bishop (v\'ease \cite{R-S} y \cite[p. 34]{Stolzenberg}). 
                \end{center} \\ 
                        \hline Si $f:(\Omega\setminus E)\subset\co\rightarrow\co$ es una funci\'on holomorfa y $E$ es un subconjunto
                        compacto de capacidad $0$, entonces $f$ es extendible a una funci\'on holomorfa
                        en la regi\'on completa $\Omega$.
                        &
                        \vspace{0.1cm}
                        Sea $U\subset\con$ un abierto acotado de $\con$ y sea $B\subset U$ un subconjunto cerrado
                        tal que $X\subset U\setminus B$ es una subvariedad de dimensi\'on pura $k$ tal que $B\subset\overline{X}$. 
                        Si $B$ tiene capacidad $0$ relativa al \'algebra de funciones anal\'iticas en $X$ que son
                        continuas en $\overline{X}$ y si existe una funci\'on $f:U\rightarrow\co^k$ propia en $B$ tal que $f(B)$ 
                        no sea un subconjunto abierto conexo de $\co^k$, entonces $\overline{X}\cap U$ es un subconjunto anal\'itico
                        de $U$ (v\'ease \cite[teorema 4]{Bishop}).\\ 
                        \hline 
                \begin{center} 
                        Teorema de compacidad de Montel. 
                \end{center} 
                        & 
                \begin{center}
                Teorema de sucesiones de variedades anal\'iticas con volumen uniformemente acotado de Bishop.
                \end{center}\\
                \hline 
                \vspace{0.1cm}
                Sea $\lbrace\Gamma_i\rbrace$ una sucesi\'on de gr\'aficas de funciones holomorfas uniformemente  
                acotadas, $f_i:\Delta\rightarrow\co$ tales que $\Gamma_i\overset{d_H}\longrightarrow\Gamma$ (convergencia de Hausdorff), 
                donde $\Gamma\subset\co^2$ es un subconjunto cerrado y $\Delta$ es el disco unitario en $\co$, 
                entonces $\Gamma$ es la gr\'afica de una funci\'on holomorfa.  
                        & 
                Sea $\lbrace V_i\rbrace$ una sucesi\'on de subvariedades anal\'iticas de $\Omega\subset\con$ con vol\'umenes 
                uniformemente acotados tales que $V_i\overset{d_H}\longrightarrow V\subset\Omega$ en el sentido de Hausdorff, entonces 
                $V$ es una subvariedad anal\'itica de $\con$ (\cite[p. 30]{Stolzenberg}). \\ \hline         
        \end{tabular} 
\end{table}

M\'as a\'un, las variedades anal\'iticas en s\'i se pueden pensar claramente como una generalizaci\'on de funciones holomorfas, 
ya que cada funci\'on holomorfa $f:\Omega\subset\con\rightarrow\co$ tiene asociada como variedad anal\'itica a su divisor cero $[Z]=Z(g)$
o como ya se mencion\'o previmente, a su gr\'afica $\Gamma_{f}\subset\Omega\times\co$, la primera es una subvariedad anal\'itica de dimensi\'on $n-1$ y la segunda es
una subvariedad anal\'itica de dimensi\'on pura $n$. Aunado a esto, es ampliamente conocido que en el contexto cl\'asico
de la variable compleja, toda funci\'on holomorfa en una variedad holomorfa compacta es constante, por ejemplo $\cp^n$ es compacta,
y por lo tanto toda funci\'on holomorfa global es constante. Sin embargo, como sabemos $\cp^n$ tiene muchas subvariedades anal\'iticas, tantas
como subconjuntos de funciones polinomiales homog\'eneas algebraicamente independientes. El cuadro \ref{Tab} enlista una serie de
versiones de teoremas cl\'asicos de an\'alisis complejo y sus correspondientes en el contexto de variedades anal\'iticas.

\section{L\'imites de hojas compactas de foliaciones en Variedades K\"ahler}
\subsection{Variedades K\"ahler, foliaciones y sus vol\'umenes}
\noindent Adem\'as de la geometr\'ia compleja y el an\'alisis complejo, otra aplicaci\'on de los teoremas de Bishop y en particular el teorema
\ref{Bishop2}, es el siguiente resultado sobre foliaciones holomorfas en variedades de K\"ahler con hojas compactas y volumen 
uniformemente acotado. 

Se define una variedad de K\"ahler como una variedad compleja $(M,I)$, es decir $M$ es una variedad diferenciable real de dimensi\'on $2d$ e 
$I$ es una estructura compleja $I^2=-1$ con una estructura adicional m\'etrica dada por $h=g-i\omega$ una m\'etrica hermitiana tal que su 
(1,1)-forma definida por $\omega=-\Im (\omega)$ sea cerrada, es decir, $d\,\omega=0$. Observamos que $g$ es una m\'etrica riemanniana con la misma 
forma de volumen que la inducida por $\omega$. Adem\'as, es importante mencionar que toda subvariedad compleja de una variedad K\"ahler es 
K\"ahler. De la misma forma, las subvariedades compactas complejas de una variedad K\"ahler minimizan el volumen en su clase de homolog\'ia. 
Adem\'as de las previas definiciones, hay que mencionar que una foliaci\'on holomorfa $\mathfrak{F}$ en una variedad compleja $(M,I)$ es una 
distribuci\'on involutiva de subespacios del haz tangente con dimensi\'on constante. Geom\'etricamente se describe a la foliaci\'on como 
una partici\'on de $M=\bigcup\mathcal{L}_z$, donde cada $\mathcal{L}_z$ es una subvariedad holomorfa de $(M,I)$ de una dimensi\'on dada.

\subsection{Estabilidad en Variedades K\"ahler y la conjetura de Beauville}

\noindent As\'i, es clara la descripci\'on del siguiente teorema, el cual demostramos Dr. Verjovsky y yo como parte del art\'iculo previamente mencionado:
\begin{teorema}[Edwards, Millet y Sullivan]\label{EMS}
        Sea $M$ una variedad compacta K\"ahler conexa de dimensi\'on compleja $n$, es decir $2n$ real, y $\mathfrak{F}$ una foliaci\'on holomorfa por hojas
        compactas de dimensi\'on real $2d$ donde $d<n$, entonces:
        \begin{enumerate}
                \item[a)] El volumen con respecto a la m\'etrica K\"ahler de las hojas es uniformemente acotado.
                \item[b)] El espacio cociente $M/\mathfrak{F}$ es un orbifold complejo con singularidades en las hojas de holonomia no trivial.
        \end{enumerate}
\end{teorema}
Aunado a esto, demostramos que la funci\'on volumen $\nu:M/\mathfrak{F}\rightarrow\re^{+}$ definida por el volumen 
\[
        \nu(\mathcal{L}_z):=\textrm{Vol}_{2d}(\mathcal{L}_z)
\] 
es discretamente semicontinua inferiormente, es decir, que para todo $n\in\zah^{+}$ y $z\in M$ se cumple que existe una vecindad
tal que para todo $\epsilon\in\re^{+}$ 
\[
        \nu(y)>n\nu(z)\hspace{0.2cm}\text{o}\hspace{0.2cm}|\nu(y)-k\nu(z)|<\epsilon\hspace{0.2cm}\text{para alg\'un}\hspace{0.2cm}k\in\{1,\dots,n\}.
\]
\noindent M\'as a\'un, los brincos en la continuidad corresponden a las hojas con holonom\'ia no trivial, las cuales son cubiertas
por hojas con holonom\'ia trivial. Como todas las hojas son compactas, la holonom\'ia es finita y el volumen de las hojas
con holonom\'ia no trivial es una fracci\'on del volumen de las hojas con holonom\'ia trivial por estabilidad de Reeb \cite{Thurston}. 
Luego, el teorema generalizado de estabilidad de Reeb \cite{Thurston} nos dice c\'omo obtener las cartas coordenadas de $M/\mathfrak{F}$. 
En el caso de las hojas con holonom\'ia trivial, para cada hoja $\mathcal{L}$ existe una vecindad saturada, es decir, $U=\bigcup_{z\in U}\mathcal{L}_z$ de la hoja, 
tal que $U$ es biholomorfa a $\mathcal{L}\times B$ donde $B\subset\co^{n-d}$ es una bola. Adem\'as, cada hoja en $U$ es biholomorfa a $\mathcal{L}\times\{w\}$ con $w\in B$. 
Esto quiere decir que $M/\mathcal{F}$ tiene una carta coordenada holomorfa a una bola de dimensi\'on compleja $n-d$ en cada hoja 
con holonom\'ia trivial. Ahora, en el caso de que la hoja $\mathcal{L}$ tenga holonom\'ia no trivial, sabemos que el grupo de holonom\'ia es 
finito y as\'i el mismo resultado de Reeb nos dice que de todas formas existe un abierto saturado $U$, el cual es un haz en 
discos de dimensi\'on complementaria a la dimensi\'on de $\mathcal{L}$, cuyo grupo de estructura es el grupo de holonom\'ia.
Es decir, que las hojas en dicho abierto son cubrientes de nuestra hoja original y que este abierto de $\mathcal{L}$ otorga a $M/\mathcal{F}$ 
una carta de orbifold complejo alrededor de este punto.
Esto se puede contrastar con el enunciado y prueba originales de Edwards, Millet y Sullivan, los cuales son m\'as generales, pero al mismo tiempo pierden 
las peculiaridades de la geometr\'ia k\"ahleriana, adem\'as el Dr. Verjovsky y yo hacemos mayor \'enfasis en la estructura del 
espacio de hojas (v\'ease \cite{EMS}). 

Es importante observar que es posible encontrar foliaciones en variedades compactas real anal\'iticas e incluso algebraicas, donde 
todas las hojas de la foliaci\'on son curvas cerradas (c\'irculos) cuyas longitudes no se encuentren uniformemente acotadas, v\'ease el ejemplo de \cite{E-V}. 
Algo que es importante destacar de este ejemplo es que la codimensi\'on es lo suficientemente grande 
para garantizar que el volumen (longitud) no se encuentre uniformemente acotado y tambi\'en sucede que la funci\'on de longitud de las hojas 
no es localmente acotada.

Lo anteriormente mencionado nos pone en el contexto de la conjetura de Beauville (2000):
\begin{conjetura}[Beauville]\label{Beauville}
        Sea $M$ una varidead compacta K\"ahler tal que exista una descomposici\'on holomorfa de su haz tangente
        \[
        TM=\bigoplus_{i\in I}\mathcal{F}_i\hspace{0.2cm}\text{tal que cada}\hspace{0.2cm}\bigoplus_{i\in J}\mathcal{F}_i\,,\,J\subset I\hspace{0.2cm}\text{es involutivo},
        \]
        entonces el cubriente univeral de $M$ es isomorfo a un producto 
        \[
        \widetilde{X}\cong\prod_{i\in I}U_i\hspace{0.2cm}\text{de tal forma que esto induce}\hspace{0.2cm}T\widetilde{X}\cong\bigoplus_{i\in I}\widetilde{\mathcal{F}}_i
        \]
\end{conjetura}
Recientemente, Druel, Pereira, Pym y Touzet demostraron una versi\'on de esta conjetura en el contexto que expusimos previamente,
pero con el enfoque particular de variedades de Poisson, v\'ease \cite{DPPT}.
\begin{teorema}[Druel, Pereira, Pym y Touzet]\label{DPPT}
        Sup\'ongase que $M$ es una variedad compacta K\"ahler tal que su haz tangente se escinde $TM=\mathfrak{F}\oplus\mathfrak{G}$, donde
        los subhaces $\mathfrak{F}$ y $\mathfrak{G}$ son involutivos. Si $\mathfrak{F}$ tiene una hoja compacta $L$ con holonom\'ia finita,
        entonces $\widetilde{M}$ es biholomorfa a un producto de variedades $N\times P$ cuyas haces tangentes son isofomorfos
        a $\mathfrak{F}$ y $\mathfrak{G}$ respectivamente.
\end{teorema}
\noindent Como observaci\'on, se puede apreciar del resultado anterior que en una variedad de K\"ahler conexa, la existencia 
de una hoja compacta con holonom\'ia finita implica que todas las hojas lo son. Esto tambi\'en lo pudimos demostrar utilizando 
el teorema \ref{Bishop2}. Este resultado ya era conocido, pero nuestra prueba es distinta a la expuesta en el art\'iculo 
original \cite{Pereira}. 
Nosotros creemos que es posible demostrar la proposici\'on de Druel, Pereira, Pym y Touzet utilizando l\'imites de las hojas 
de forma similar a la previamente expuesta. M\'as sobre esto se expondr\'a en la secci\'on de \textit{Problemas a resolver}.

Es claro por lo que se ha mostrado aqu\'i que existe un v\'inculo importante entre la estructura de un espacio anal\'itico y su volumen.
En el caso de las foliaciones, podemos extender esta noci\'on a los vol\'umenes de sus hojas, adem\'as de que los resultados de Bishop 
nos otorgan un puente entre geometr\'ia y an\'alisis, por lo que proponemos  estudiar v\'inculos m\'as profundos entre estas
dos \'areas utilizando las herramientas previamente expuestas adem\'as de otros m\'etodos de la geometr\'ia compleja moderna.
Como ya fue aludido con anterioridad, las foliaciones holomorfas en variedades de tipo K\"ahler son de particular inter\'es en este aspecto
y por lo tanto es en este contexto que pensamos que la expansi\'on de nuestra investigaci\'on podr\'ia ser m\'as fruct\'ifera en la
b\'usqueda de resultados nuevos.

\section{Comportamiento local de foliaciones de hojas uniformizadas por discos y sus l\'imites}

\noindent En t\'erminos de comportamiento local de las foliaciones holomorfas, Dr. Verjovsky y yo hemos estudiado el caso de foliaciones por 
curvas complejas (superficies de Riemann), en la bola unitaria $B\subset\co^n$, en el caso cuando se encuentran uniformizadas por el 
disco unitario $\Delta:=\{|z|< 1\}\subset\co$, es decir las hojas tienen una estructura hiperb\'olica. As\'i, se busca el comportamiento 
l\'imite de las hojas si se piensa al disco como una variedad K\"ahler completa. En este caso, las hojas no son necesariamente compactas 
ni es necesario que su volumen sea acotado. Se busca saber m\'as acerca de las propiedades m\'etricas y geom\'etricas de las hojas. 
En este sentido, se entiende por \emph{l\'imite} al conjunto al que se aproximan las geod\'esicas de las hojas en $\overline{B}$ cuando las recorremos 
en un tiempo infinito.
Para entender estos l\'imites de geod\'esicas, proponemos estudilarlos por medio de una compactaci\'on del disco adecuada donde 
sea posible esclarecer algo sobre la pregunta en cuesti\'on. Con esto en mente, proponemos encajar al disco en el espectro 
del \'algebra de Banach de funciones holomorfas acotadas en el disco, es decir, $\mathcal{H}^{\infty}=\mathcal{L}(\Delta)^{\infty}\cap\mathcal{H}(\Delta)$, 
as\'i \hbox{$M^{\infty}:=\{\kappa:\mathcal{H}\rightarrow\co\,|\,\kappa(fg)=\kappa(f)\kappa(g),\,\kappa(f+g)=\kappa(f)+\kappa(g),\,\kappa(1)=1\}$} y por lo tanto $\Delta$ claramente encaja 
en $M^{\infty}$, $\Delta\rightarrow M^{\infty}$ por medio de la funcion evaluaci\'on $ev(z)[f]=f(z)$. 
Se saben muchas propiedades del conjunto $M^{\infty}$, es un espacio m\'etrico compacto con la topolog\'ia d\'ebil dada por 
la acci\'on natural de $\mathcal{H}^{\infty}$ en $M^{\infty}$, es decir para toda $f\in\mathcal{H}^{\infty}$ define $\tilde{f}:M^{\infty}\rightarrow\co$ dada por $\kappa\mapsto\kappa(f)$ (v\'ease \cite{SCHARK}). 
Adem\'as, al actuar la funci\'on identidad $id_{\Delta}$ en $M^{\infty}$, se obtiene lo siguiente $\pi=\tilde{id_{\Delta}}$ 
es la inversa al encaje $ev:\Delta\rightarrow M^{\infty}$, claramente. 
M\'as a\'un como $M^{\infty}$ es compacta, la imagen bajo $\pi$ es un conjunto compacto. Se puede demostrar que este conjunto es en efecto
$\overline{\Delta}\subset\co$ (v\'ease \cite{SCHARK}). De esta forma, al investigar el subconjunto $M^{\infty}\setminus\Delta$, conocido como 
la \emph{corona} se obtienen muchas propiedades interesantes; este conjunto fibra sobre $S=\overline{\Delta}\setminus\Delta$ por medio de el mapeo $\pi$ 
y por lo tanto es posible extender cualquier funci\'on continua definida en $\Delta$ a una funci\'on en $M^{\infty}$, la cual es
constante en las fibras de $\pi$ sobre $S$.
Adem\'as, los biholomorfismos de $\Delta$ act\'uan de forma natural en $\mathcal{H}^{\infty}$ por medio de la composici\'on derecha, por lo que es posible 
definir a los biholomorfismos de $\Delta$ en $M^{\infty}$. Simplemente si $\varphi$ es un biholomorfismo, entonces $\varphi^{*}[\kappa](f)=\kappa(f\circ\varphi)$ est\'a 
bien definida y es un homeomorfismo de $M^{\infty}$. 
Con esto es posible demostrar que las fibras de $\pi$ en $S$ son homeomorfas entre s\'i bajo 
rotaciones por el origen y adem\'as notamos que esto permite actuar al grupo de isometr\'ias hiperb\'olicas $\pslr$ en $M^{\infty}$,
por lo que es posible hablar de cierta estructura hiperb\'olica en $M^{\infty}$. As\'i, definimos el l\'imite como la imagen $\overline{\varphi}(M^{\infty}\setminus\Delta)$, donde $\varphi$ es la uniformizaci\'on
de la hoja y $\overline{\varphi}$ es su extensi\'on a $M^{\infty}$. El contexto de estas preguntas que hemos hecho es: 
\begin{enumerate}
        \item Toda foliaci\'on en la bola $B\subset\co^2$ con una singularidad en el origen tiene una hoja que pasa por el origen
        por el teorema de Camacho-Sad. As\'i, en $B\setminus\{0\}$ las hojas son incompletas y por lo tanto existe una hoja
        y una geod\'esica cuyo l\'imite es $0$ (v\'ese \cite[teorema 3.3]{brunella}).
        \item Sorprendentemente, existen foliaciones no singulares por discos en $B$ donde las hojas son completas, por lo que 
        encontrar los l\'imites de las geod\'esicas en la bola $\overline{B}$ es interesante en esos casos, v\'ease \cite{A-F}.
\end{enumerate}

\section{Variedades de Fano-Poisson}
\noindent  Inspirados por la demostraci\'on del teorema \ref{DPPT} del art\'iculo reciente \cite{DPPT}, creemos que existen 
relaciones interesantes entre la funci\'on de volumen de las hojas definidas por la foliacion natural de una variedad \emph{Poisson}
(teorema \ref{weins}) y la estructura de una variedad Fano-Poisson.
\subsection{Variedades de Poisson, el teorema de Weinstein y su foliaci\'on natural}
\noindent Primero hay que recordar que una variedad compleja $M$ es de Poisson si existe una operaci\'on bilineal en el anillo de g\'ermenes de 
funciones holomorfas en $M$, la cual se le conoce por el nombre de \emph{corchete de Poisson}. Denotaremos por $\mathcal{O}_M:=\mathcal{H}_M/\sim$, donde $f_1\sim f_2$ si 
$f_1=f_2|_U$, al anillo de g\'ermenes de funciones holomorfas. As\'i, un corchete de Poisson es una funci\'on bilineal
\[
\{\cdot,\cdot\}:\mathcal{O}_M\times\mathcal{O}_M\rightarrow\mathcal{O}_M,
\]
que cumple las siguientes propiedades
\begin{enumerate}
\item $\{f,g\} = -\{g,f\}$
\item $\{f,gh\}=\{f,g\}h + g\{f,h\}$
\item $\{f,\{g,h\}\}+\{g,\{h,f\}\} + \{h,\{ f,g\}\}=0$.
\end{enumerate}
\noindent Se observa que un germen de una funci\'on fija $H\in\mathcal{O}_M$ define un campo vectorial definido por $\xi_H(\cdot)=\{H,\cdot\}$.
A dicho campo vectorial le llamamos el campo \emph{hamiltoniano} definido por $H$. Utilizando la notaci\'on de campos \emph{multivectoriales}, 
una definici\'on alternativa del corchete de Poisson es la siguiente: al denotar los campos vectoriales holomorfos como $\mathcal{T}M=H^0(M,TM)$,
entonces el espacio de \emph{$p$-vectores} se define por 
\[
        \Lambda^{p}(\mathcal{T}M):=\{\mathcal{O}_M\times\dots\times\mathcal{O}_M\rightarrow\mathcal{O}_M\,\vert\,\text{anti sim\'etrica}\}.
\]
Entonces, un corchete de Poisson es un campo \emph{bivectorial}, el cual podemos definir por medio del emparejamiento $\langle\cdot,\cdot\rangle$,
entre los campos $p$-vectoriales y el espacio de \emph{$p$-formas diferenciales holomorfas} $\Omega^{p}(M)$ por medio de
\[
        \pi\in H^0(M,\Lambda^2(\mathcal{T}M))\,\text{ entonces }\,\{f,g\}=\langle \pi,df\wedge dg\rangle.
\]
Por lo tanto, generalizando esto tenemos el siguiente mapeo definido por un corchete de Poisson $\pi$
\[
        \pi^{\#}:\Omega^1_M\rightarrow\mathcal{T}M,\hspace{0.2cm}\pi^{\#}(\alpha):=\iota_{\alpha}(\pi):=\langle\pi,\alpha\wedge\cdot\rangle.
\]
\noindent Con esto, el rango de $\pi$ en un punto $p\in M$ se define como el entero positivo $r\in \zah^{+}$ que define la dimensi\'on del
espacio m\'aximo, donde $\pi^{\#}_p$ es no degenerada, es decir, es la dimensi\'on del espacio m\'as grande, tal que $\pi^{\#}$ es una
biyecci\'on. Si $\pi^{\#}$ es no degenerada, su rango es $2n=\dim(M)$, entonces $\pi^{-1}$, la inversa de $\pi^{\#}$, define una forma simpl\'ectica
en $M$. El teorema de escici\'on de Weinstein define una foliaci\'on natural en una variedad $(M,\pi)$ de Poisson, pero primero 
recordamos que una funci\'on holomorfa entre dos variedades de Poisson $(M_1,\{\cdot,\cdot\}_1)$ y $(M_2,\{\cdot,\cdot\}_2)$ es un morfismo de Poisson 
$\phi:M_1\rightarrow M_2$ si $\{f,g\}_1\circ\phi=\{f\circ\phi,g\circ\phi\}_2$.
\begin{teorema}[Weinstein]\label{weins}
        Sea $(M,\pi)$ una variedad holomorfa de Poisson de dimensi\'on real $2n$. Supongamos que $\pi$ es de rango $2r$ en un punto $x\in M$, 
        entonces existe una vecindad $U$ de $x$ tal que $U$ es isomorfa en el sentido de Poisson a un producto $S\times P$ de tal forma que $S$ es
        simpl\'ectica con coordenadas $(p_i,q_i)_{i=1}^r$ y $(P,\tilde{\pi})$ es una variedad de Poisson de rango cero en $x$
        con coordenadas $z=(z_j)_{j=1}^{2n-2r}$
        \[
                \pi=\sum_{i=1}^r \partial{p_i}\wedge\partial{q_i}+\sum_{1\leq j\leq k\leq 2n-2r} f^{jk}(z)\partial{z_j}\wedge\partial{z_k}.
        \]
\end{teorema} 
\noindent Observamos del teorema anterior que $f^{jk}(x)=0$. Este teorema define claramente una foliaci\'on natural en $(M,\pi)$ de hojas
simpl\'ecticas. Sin embargo, no todas son de las misma dimensi\'on, por lo que es necesario notar que la definici\'on de foliaci\'on
se puede expandir a este contexto m\'as general, es decir, una foliaci\'on es simplemente un $\mathcal{O}_M$-subm\'odulo de $\mathcal{T}M$ involutivo.
Ahora, esto define una filtraci\'on $X_0\subset X_2\subset X_4\subset\dots\subset M$, donde $X_{2k}=\{x\in M\,|\,\textrm{rang}(\pi_x)\leq 2k\}$,
si denotamos a las hojas simpl\'ecticas de $(M,\pi)$ como $\mathcal{L}$, entonces tambi\'en podemos pensar a $X_{2k}$ como 
$$
X_{2k}=\bigcup_{\dim(\mathcal{L})\leq 2k}\mathcal{L}.
$$
\subsection{Variedades de Fano-Poisson desde la perspectiva de la geometr\'ia k\"ahleriana}
\noindent Con esto establecido, presentaremos una conjetura establecida en 1993 por Bondal para variedades Fano-Poisson que esperamos poder 
iluminar con nueva informaci\'on utilizando la funci\'on volumen. Una variedad $M$ es Fano si cumple lo siguiente 
(v\'ease \cite{S-Yau} y \cite{ZB}):
\begin{itemize}
        \item $M$ admite una m\'etrica de K\"ahler-Einstein, es decir, si definimos la m\'etrica por medio de su forma simpl\'ectica 
        K\"ahler $\omega$, entonces
                $$\textrm{Ric}_{\omega}=\lambda\omega\hspace{0.2cm}\lambda\in\re,$$
        donde en coordendas podemos calcular la curvatura de Ricci para la m\'etrica 
        \hbox{$h=g-i\omega=\sum h_{i\overline{j}}\,dz_i\otimes d\overline{z}_j$} como
        $$\textrm{Ric}_{\omega}=\frac{i}{2\pi}\sum_{ij}R_{i\overline{j}}\,dz_i\wedge d\overline{z}_j=\frac{-i}{2\pi}\partial\overline{\partial}\log(\det(h_{k\overline{l}})).$$
        \item La clase de cohomolog\'ia definida por la curvatura de Ricci (primera clase de Chern) es positiva
        $$
        [\textrm{Ric}_{\omega}]=c_1(M)>0.
        $$
\end{itemize}
\noindent Hay que mencionar que normalmente en la literatura se define una variedad Fano como una variedad algebraica $X$ completa 
en el sentido de que toda proyecci\'on $X\times Y\rightarrow Y$ es cerrada para toda variedad algebraica $Y$, tal que su divisor/haz antican\'onico
$K^{*}_{X}$ es amplio, por lo tanto, toda variedad Fano es \emph{proyectiva}. La disparidad entre la definici\'on que dimos y la utilizada 
ampliamente en la literatura proviene de la demostraci\'on dada por Yau de la conjetura de Calabi (v\'ease \cite{S-Yau}). Una varieda
\emph{Fano-Poisson} es entonces una variedad compleja con estas dos estructuras, en particular son compactas K\"ahler \cite{Myers}. 
\begin{conjetura}[Bondal]\label{Bondal}
         Sea $M$ una variedad de Fano-Poisson, bajo la notaci\'on previamente establecida, $X_{2k}$ es la uni\'on de las hojas 
         simpl\'ecticas de dimensi\'on $2k$, entonces $X_{2k}$ tiene una componente de dimensi\'on mayor a $2k$.
\end{conjetura}
\noindent No sabemos si es cierta o falsa o si es posible demostrar con lo que vamos a proponer, pero creemos que como una 
investigaci\'on entre la relaci\'on entre la geometr\'ia complejo-diferencial y la geometr\'ia algebraica lo que proponemos es interesante. 
Esta conexi\'on creemos que es relevante para el acercamiento a la geometr\'ia que nosotros proponemos, esto est\'a inspirado por trabajos 
previos de Yau \cite{S-Yau} y sobre todo Donaldson y Sun \cite{D-SS}. En particular, Donaldson y Sun muestran resultados de naturaleza muy similar 
a los resultados de Bishop (teorema \ref{Bishop2}) en el contexto generalizado de l\'imites de Gromov-Hausdorff en variedades de 
K\"ahler, con particular aplicaci\'on a las variedades de Fano.

\section{Especulaciones y problemas a resolver}
\begin{itemize}
%\textcolor{red}{\centerline{Problema a resolver, hip\'otesis:}}
        \item En el caso de la conjetura de Beauville \ref{Beauville}, pretendemos primero demostrar el caso en el que 
        las hojas sean compactas, utilizando los m\'etodos de Bishop. As\'i obendr\'iamos una demostraci\'on alternativa al
        teorema \ref{DPPT}.

        \item  Creemos que es posible prescindir de la suposici\'on de hojas compactas suponiendo solamente que las 
        hojas son de volumen localmente acotado, es decir, que las hojas son de volumen finito y en cualquier punto existe
        una vecindad tal que la foliaci\'on en esa vecindad son hojas con volumen uniformemente acotado. Bajo esta 
        suposici\'on podemos garantizar que el espacio de hojas es un espacio anal\'itico complejo, adem\'as la foliaci\'on es 
        localmente una fibraci\'on (v\'ease \cite{A-V}). 
        
        \item Bajo estas suposiciones, creemos que es posible definir un biholomorfismo entre los cubrientes universales 
        de las hojas por medio de levantamientos de curvas y trabajar con el grupoide de holonomia de la foliaci\'on 
        similar a lo realizado en \cite{DPPT}. Adem\'as, creemos que es posible extraer informaci\'on m\'etrica de las
        hojas (o su cubriente) si se levantan geod\'esicas en lugar de curvas arbitrarias, donde simplemente utilizamos la 
        m\'etrica riemanniana en $M/\mathfrak{F}$ heredada de la m\'etrica K\"ahler original. 
        
        \item En el caso de foliaciones con volumen acotado, especulamos que el fen\'omeno de semicontinuidad que se expuso en el caso
        de los vol\'umenes, sucede de manera similar a nivel de grupos de homotop\'ia y que posiblemente nos permita comprender m\'as sobre la estructura 
        anal\'itica del cubriente universal de las hojas, el cual en el caso de variedades K\"ahler es el mismo para todas \'estas
        si es v\'alida la hip\'otesis de Beauville. Es decir, en el l\'imite de una sucesi\'on de hojas con volumen acotado, su cubriente
        universal es el mismo y el grupo fundamental del l\'imite tiene como subgrupo al de las hojas que se le aproximan,
        posiblemente sea necesario pedir que dichos grupos sean finitos.

        \item Otorgar un significado homol\'ogico/cohomol\'ogico a la semicontinuidad discreta en el teorema \ref{EMS}.

        \item Si suponemos que el espacio $M$ o el espacio de las hojas $M/\mathfrak{F}$ tienen estructura adicional (Poisson, Fano, Stein, proyectiva) 
        ¿Qu\'e tipo de comportamiento pueden tener las funciones de volumen de una variedad?

%\textcolor{red}{\centerline{Especulamos:}}
        \item Si $\mathfrak{F}$ es una foliaci\'on con una singularidad aislada en el origen de $B$ cuyas hojas se uniformizan por el disco $\Delta$,
        entonces existe una hoja incompleta y una geod\'esica en \'esta cuyo l\'imite es $0$.  

%\textcolor{red}{\centerline{Proponemos:}}
        \item Estudiar el crecimiento del volumen de los conjuntos $X_{2k}\cap B_{R}(x)$ donde $B_R(x)$ es la bola m\'etrica 
        de radio $R$ centrada en un punto $x\in X_{2k}$ cuando $R\rightarrow\infty$. Si el conjunto $X_{2k}$ es 
        de dimensi\'on mayor a $2k$, esperamos un comportamiento de crecimiento de volumen mayor a $\mathcal{O}(R^{2k})$.
        
        \item Buscar casos en los que existan descripciones de los conjuntos $X_{2k}$ que nos permitan describir dichos conjuntos
        como l\'imites de Gromov-Hausdorff y hacer uso de cotas dadas en \cite{D-SS} para determinar la tasa de crecimiento
        del volumen de las hojas simpl\'ecticas.

        \item Determinar si la condici\'on de no colapso del volumen dada en \cite{D-SS} y las estimaciones anal\'iticas similares
        nos permiten hacer c\'alculo del volumen de subvariedades en variedades Fano. Si esto es posible, encontrar
        qu\'e tipo de propiedades tiene la funci\'on volumen en una foliaci\'on de una variedad Fano.

        \item Hacer un contraste entre los aspectos geom\'etricos encontrados y las demostraciones a la conjetura \ref{Bondal} en los
        casos ya demostrados como en \cite{Gua-Pym}.
\end{itemize}
\begin{thebibliography}{3}

\bibitem{A-F} Alarc\'on, A., Forstneric (2019) \textit{A foliation of the ball by complete holomorphic discs}, Mathematische Zeitschrift
pp. 169-174, Springer-Verlag.

\bibitem{A-V} Alexander J.C., Verjovsky A., (1988) \textit{First Integrals for Singular Holomorphic Foliations With Leaves of Bounded Volume}
Holomorphic Dynamics. Lecture Notes in Mathematics, vol 1345. Springer, Berlin, Heidelberg.

\bibitem{Beuville1} Beauville, A. (2000) \textit{Complex manifolds with split tangent bundle}, Complex analysis and algebraic geometry
, de Gruyer, Berlin.

\bibitem{ZB} Blocki, Z., (2011) \textit{The Complex Monge-Amp\`ere Equation in K\"ahler Geometry} Lecture Notes in Mathematics 2075
paart of Pluripotential Theory, Springer-Verlag, pp. 95-143.

\bibitem{Bishop} Bishop, E. (1964) \textit{Conditions for the Analyticity  of certain sets}, Michigan Math. J. 11, No. 4, 289--304. 

\bibitem{Bondal} Bondal, A. I,. (1993) \textit{Non-commutative deformations and Poisson brackets on projetive spaces}, 
Max-Planck-Institut f\"ur Matematik, Germany.

\bibitem{brunella} Brunella, M., (2015) \textit{Birational Geometry of Foliations}, IMPA Monographs, 
Springer International Publishing, Switzerland.

\bibitem{Chirka} Chirka E. M. (1989) \textit{Complex Analytic Sets}, Kluwer
Academic, Dordrecht, The Netherlands. 

\bibitem{Chow} Chow, W-L. (1949) \textit{On Compact Complex Analytic Varieties},
American Journal of Mathematics, Vol. 71, No. 4, pp. 893-914.

\bibitem{D-SS} Donaldson, S., Sun, S. (2014) \textit{Gomov-Hausdorff Limits of K\"ahler Manifolds and Algebraic Geometry I},
 Acta Math 213, 63–106, Springer-Verlag.

\bibitem{DPPT}Druel S., Pereira J V., Pym B., Touzet F. \textit{A global Weinstein splitting theorem for 
holomorphic Poisson manifolds}, Por publicarse.

\bibitem{EMS} Edwards, R., Millet, K., Sullivan, D. (1975) \textit{Foliations
With All Leaves Compact}, Topology Vol. 16, pp. 13-32. Pergamon Press, 1977.

\bibitem{V-A} Epstein, D. B. A., Millet, K. C., Tischler D.
(1977) \textit{Leaves Without Holonomy}, Journal of the London Mathematical
Society.

\bibitem{E-V} Epstein, D. B. A., Voght, E. (1978) \textit{A Counterexample to the Periodic Orbit Conjecture}, 
Annals of Mathematics, Vol. 108, pp. 539-552. 

\bibitem{Epstein1} Epstein, D. B. A. (1976) \textit{Foliations with all leaves compact}, Annales de l'Institute Fourier, 
tome 26 no. 1, pp. 265-282.

\bibitem{Epstein2} Epstein, D. B. A. (1977) \textit{Periodic Flows on Three-Manifolds}, Annals of Mathematics, 
Second Series, Vol. 95, No. 1 (Jan., 1972), pp. 66-82.

\bibitem{Gua-Pym} Gualtieri, M., Pym, B. (2012) \textit{Poisson modules and degeneracy loci},
Proceedings of the London Mathematical Society 107(3).

\bibitem{R-S} Remmert R., Stein, K. (1953) \textit{Über die wesentlichen
Singulariäten analyscher Mengen}. Math. Annalen, Bd. 126, S. 263--306.

\bibitem{G-R} Robert C. (1965) \textit{Analytic Functions of Several
Complex Variables}, Prentice-Hall, Inc, Englewood Cliffs, N.J.

\bibitem{Rudin} Rudin W. (1980) \textit{Function Theory in the Unit Ball of
$\con$}, Springer-Verlag Berlin Heidelberg.

\bibitem{GAGA} Serre, J.-P. (1956) \textit{G\'eom\'etrie alg\'ebrique et
g\'eom\'etrie analytique}, Annales de l'institut Fourier, Vol. 6. 

\bibitem{Myers} Myers, S. B. (1941), \textit{Riemannian manifolds with positive mean curvature}, Duke Mathematical Journal, 
8 (2): 401–404.

\bibitem{SCHARK} Schark, I. J. \textit{Maximal Ideals in an Algebra of Bounded Analytic Functions}, The Institute for Advanced Study
Princeton, New Jersey.

\bibitem{Stolzenberg} Stolzenberg G. (1966) \textit{Volumes, Limits and
Extensions of Analytic Varieties}, Lecture Notes in Mathematics,
Springer-Verlag, Berlin. 

\bibitem{Thurston} Thurston W., P. (1974) \textit{A Generalization of Reeb Stability Theorem}, Topology Vol. 13.pp. 347-352.
Pergamon Press, Great Britain.

\bibitem{Pereira} Pereira J. V., (2001) \textit{Global Stability for Holomorphic Foliations in Kaehler Maniforlds}. 
Qualitative Theory of Dynamical Systems volume 2, pages 381–384.

\bibitem{S-Yau} Yau, S-T., (1978) \textit{On The Ricci Curvature of a Compact K\"ahler Manifold and the Complex Monge-Amp\`ere Equation, I*},
Communications on Pure and Applied Mathenatics, Vol. XXXXI, 339-411, John Wiley \& Sons. Inc.
\end{thebibliography}

\end{document}
