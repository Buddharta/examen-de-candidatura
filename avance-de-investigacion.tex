
\documentclass[letterpaper]{report}
\usepackage{pst-node}
\usepackage{tikz-cd} 
\usepackage{amsmath}
\usepackage{float}
\usepackage{amsfonts}
\usepackage{amssymb}
\usepackage[spanish,activeacute]{babel}
\usepackage{amscd}
\usepackage{graphicx}
\usepackage{color}
\usepackage{transparent}
\graphicspath{{./figs/}}
\usepackage{makeidx}
\usepackage{afterpage}
\usepackage{array}
%%\makeindex

\newtheorem{teorema}{Teorema}[chapter]
\newtheorem{prop}[teorema]{Proposici\'on}
\newtheorem{cor}[teorema]{Corolario}
\newtheorem{lema}[teorema]{Lema}
\newtheorem{def.}{Definici\'on}[chapter]
\newtheorem{afir}{Afirmaci\'on}
\newtheorem{conjetura}{Conjetura}

\renewcommand{\figurename}{Figura}
\renewcommand{\chaptername}{\Large  \sc Cap\'{\i}tulo}
\renewcommand{\indexname}{\'{I}ndice anal\'{\i}tico}
\newcommand{\zah}{\ensuremath{ \mathbb Z }}
\newcommand{\nat}{\ensuremath{ \mathbb N }}
\renewcommand{\bibname}{Bibliograf\'{\i}a}
\newcommand{\dem}{{\sc Demostraci\'on. }}
\newcommand{\bg}{\ensuremath{\overline \Gamma}}
\newcommand{\ga}{\ensuremath{\Gamma}}
\newcommand{\fb}{\ensuremath{\overline f}}
\newcommand{\la}{\ensuremath{\lambda}}
\newcommand{\La}{\ensuremath{\Lambda}}
\newcommand{\bt}{\ensuremath{\overline T}}
\newcommand{\li}{\ensuremath{\mathbb{L}}}
\newcommand{\ord}{\ensuremath{\mathbb{O}}}
\newcommand{\bs}{\ensuremath{\mathbb{S}^1}}
\newcommand{\co}{\ensuremath{\mathbb C }}
\newcommand{\con}{\ensuremath{\mathbb{C}^n}}
\newcommand{\cp}{\ensuremath{\mathbb{CP}}}
\newcommand{\rp}{\ensuremath{\mathbb{RP}}}
\newcommand{\re}{\ensuremath{\mathbb R }}
\newcommand{\hc}{\ensuremath{\widehat{\mathbb C} }}
\newcommand{\pslz}{\ensuremath{PSL(2,\mathbb Z) }}
\newcommand{\pslr}{\ensuremath{PSL(2,\mathbb R) }}
\newcommand{\pslc}{\ensuremath{PSL(2,\mathbb C) }}
\newcommand{\qed}{\ensuremath{\hspace*{0em plus 1fill}\blacksquare}}
\newcommand{\hd}{\ensuremath{\mathbb H^2}}


\begin{document}


\begin{titlepage}
\begin{center}

\LARGE\textbf{ UNIVERSIDAD NACIONAL AUT'ONOMA DE MEXICO}
\vspace*{0.3cm}

\large PROGRAMA DE MAESTR'IA Y DOCTORADO EN CIENCIAS MATEM'ATICAS Y DE LA ESPECIALIZACI'ON EN ESTAD'ISTICA APLICADA
\vspace*{2cm}

\LARGE\textbf{AVANCE DE INVESTIGACI'ON (EXAMEN DE CANDIDATURA)}
\vspace*{2cm}

\small\textbf{PRESENTA:}
\vspace*{0.2cm}

\small CARLOS EDUARDO MART'INEZ AGUILAR
\vspace*{0.2cm}

\small DIRECTOR: DR. SANTIAGO ALBERTO VERJOVSKY SOL'A
\vspace*{0.2cm}

\small \textbf{INSTITUTO DE MATEM'ATICAS UNAM}
\vspace*{2.5cm}
%\tableofcontents

\end{center}
\end{titlepage}
\noindent Presento a continuaci\'on el avance durante los primeros cinco semestres de mi precandidatura a estudiante doctoral.
Durante el periodo indicado tanto yo como mi director de tesis (Dr. Santiago Alberto Verjovsky Sol\'a) nos dedicamos a estudiar 
foliaciones holomorfas $\mathfrak{F}$ cuyas hojas cumplan con ciertas propiedades geom\'etricas que son \'utiles para estudiar el espacio de hojas,
estas propiedades nos permiten dar cierta regularidad al espacio de hojas, es decir, como normalmente el espacio de hojas es 
topologicamente complejo, es necesario que las hojas de la foliaci\'on cumpla con m\'as propiedades que permitan que por lo menos
podamos asegurar que el espacio de hojas sea Hausdorff. Un ejemplo es estudiar foliaciones cuyas hojas tengan vol\'umenes acotados 
en variedades de K\"ahler $(M,h)$. La base historica de estudiar foliaciones con volumen acotado, proviente de trabajos previos hechos por matem\'aticos como
Epstein, Edwards, Millet, Reeb, Sullivan, Haeflinger, J.C Alexander y el mismo Dr.Verjovsky entre otros \cite{Epstein1}, \cite{Epstein2}, \cite{EMS}, \cite{E-V}, \cite{V-A}. 
Sin embargo la investigaci\'on que hemos propuesto hasta ahora se encuentra apoyada por el trabajo previo de J.C Alexander y el 
Dr.Verjovsky \cite{V-A} el cual a su vez se basa en trabajos de Erret Bishop \cite{Bishop} sobre extensiones y l\'imites de Hausdorff 
de sucesiones de variedades anal\'iticas y sigue el esp\'iritu de la presentaci\'on de los resultados de Bishop que se presentan en el 
Libro ``Volumes, Limits and Extensions of Analytic Vaieties'' \cite{Stolzenberg}.

M\'as precisamente hemos estado investigando maneras de extender el rango de las posibles aplicaciones de estos resultados a
distintas \'areas del an\'alisis complejo y la geometr\'ia compleja con \'enfasis en la teor\'ia de foliaciones holomorfas. En el trabajo
ya realizado hemos encontrado conecciones entre teoremas conocidos del an\'alisis complejo y la geometr\'ia compleja e incluso hemos
escrito un pequeño art\'iculo de dibulgaci\'on con estos hallazgos, el cual por el momento se encuentra en una etapa de correcciones
y posibles expansiones. Una de los primeros hallazgos de estas conecciones es por ejemplo una pueba novedosa de el teorema
\begin{teorema}[Chow]\label{Chow}
        Todo subespacio anal\'itico del espacio proyectivo complejo $\cp^{n}$ es algebraico.
\end{teorema}
\noindent Esto a partir de los resultados de Bishop sobre variedades anal\'iticas de dimensi\'on pura $k$. Aqu\'i entendemos como espacio 
anal\'itico o variedad ana\'itica a un espacio topol\'ogico Hausdorff $X$ paracompacto con una estructura anillada $\mathcal{H}_X$, es decir una 
gavilla de anillos en $X$ que es localmente isomorfa a un subconjunto anal\'itico de $\con$, es decir que para cada $x\in X$
existe una vecindad abierta $V$ y una homeomorfismo $\varphi_V:V\rightarrow Z,$ donde $U\subset\con$ es un abierto y $Z=Z(f_1,\dots,f_k)(U)$ reprecenta al 
conjunto anal\'itico de los ceros de las funciones holomorfas $f_i:U\rightarrow\co$ y el pullback $\varphi^{*}:\mathcal{H}(U)/\mathcal{I}(Z)\rightarrow \mathcal{H}(V)$ 
es un isomorfismo de anillos entre $\mathcal{H}_X(V)$ y el anillo cosiente de
\[
        \mathcal{H}(U)=\{f:U\rightarrow\co\,|\,\text{f es holomorfa}\}\hspace{0.2cm}\text{con}\hspace{0.2cm}\mathcal{I}(Z)=\{f\in\mathcal{H}(U)\,|\,f|_{Z}=0\}.
\]
As\'i un subespacio anal\'itico $Y$ de $X$ es un espacio anillado $(Y,\mathcal{H}_Y)$ con un encaje cerrado $\iota:Y\rightarrow X$ tal que localmente el anillo
$\mathcal{H}_Y$ es isomorfo por medio del pullback $\iota^{*}$ al anillo cosiente de $\mathcal{H}$ con el ideal de funciones nulas en la imagen de $Y$. 
Similarmente la definici\'on de conjunto algebraico se obtiene reemplazando el anillo de funciones anal\'iticas $\mathcal{H}(U)$ con el anillo de 
polinomios en $\con$ denotado por $\co[z_1,\dots,z_n]$.

Siguiendo la presentacion del libro de Stolzenber \cite{Stolzenberg} el resultado que liga 
los resultados de Bishop con el teorema de Chow es un resultado que estipula
\begin{teorema}[Bishop]\label{Bishop1}
        Sea $X\subset\con$ una subvariedad anal\'itica de dimensi\'on pura $k$ (compleja), si para toda $R\in\re^+$
        $$Vol_{2k}(X\cap B_R(0))\leq CR^{2k},$$
        donde $B_R(0)$ denota a la bola centrada en el origen de radio $R$ y $Vol_{2k}$ es el volumen de dimensi\'on $2k$. Entonces $X$ es algebraica.
\end{teorema}

Esta conexion entre analisis y geometría compleja nos llev\'o a explorar m\'as sobre las posibles conexiones entre estas dos ramas,
adem\'as de otros resultados con el an\'alisis complejo, otro resultado cl\'asico que puede desmostrar a partir de los resultados de Bishop
es
\begin{teorema}[Montel]\label{Montel}
        Sea $B\subset\con$ la bola unitaria, si $\mathcal{H}(\overline{B})$ es el \'algebra de Banach de funciones holomorfas que son continuas 
        en la fontera de $B$, entonces toda familia de funciones localmente acotadas $F\subset\mathcal{H}(\overline{B})$ es una familia \textit{normal}.
\end{teorema}
Recordamos que una familia de funciones $F$ es \textit{normal} si y s\'olo si su cerradura es secuencialmente compacta. Si vemos a este 
teorema desde la perpectiva de los teoremas de Bishop, entonces podemos considerar que el siguiente teorema es una generalizaci\'on en
un conexto m\'as general.
\begin{teorema}\label{Bishop2}
        Sea $\{ V_n \}_{n\in\nat}$ una sucesi\'on de variedades anal\'iticas de dimensi\'on pura $k$ en $\con$ con volumen uniformemente
        acotado por un una constante $C\in\re^{+}$, entonces si $V_n\rightarrow V$ converge a un cerrado $V\subset\con$ en el sentido de Hausdorff, entonces 
        $V$ es una variedad anal\'itica.
\end{teorema}
Convergencia de una sucesi\'on de conjuntos $S_n\rightarrow S$ en el sentido de Hausdorff para conjuntos cerrados de un espacio m\'etrico $(X,d)$
sucede cuando $S_n\cap K\rightarrow S\cap K$ en el sentido de la m\'etrica de Hausdorff $d_H$ para todo $K\subset X$ compacto, donde
\[
        d_H(K_1,K_2):= \max_{x\in K_1}\left\{d(x,K_2)\right\}+\max_{y\in K_2}\left\{d(y,K_1)\right\}. 
\]
Nosostros demostramos el resultado de Montel se demuestra por medio del teorema \ref{Bishop2} utilizando las gr\'aficas de las
funciones 
\[
        \Gamma_{f}=\{(z,w)\in B\times\co\,|\,w=f(z)\}=Z(w-f(z)),
\]
\noindent donde para una funci\'on holomorfa $g\rightarrow\co$ denotamos a su conjunto nulo como $Z(g)=\{z\in\Omega\,|\,g(z)=0\}$. Ahora aqu\'i 
pensamos a $\Gamma_{f}$ como variedades anal\'iticas de dimensi\'on pura $n$ en un abierto de $\co^{n+1}$, en el art\'iculo demostramos 
que para toda susesi\'on $\{f_n\}_{n\in\nat}\subset F$ de funciones en una familia localemente acotada, tiene una subsucesi\'on 
cuyas gr\'aficas convergen en el sentido de la m\'etrica de Hausdorff, adem\'as de demostrar que su conjunto l\'imite 
es a su vez la gr\'afica de una funci\'on holomorfa en $B$.

Como hemos visto los resultados de Bishop se pueden pensar como generalizaciones de resutados cl\'asicos del an\'alisis complejo
pero vistos dentro de la teor\'ia de variedades anal\'iticas, las cuales se pueden pensar claramente como una generalizaci\'on 
ya que cada funci\'on holomorfa en un abieto $f:\Omega\subset\con\rightarrow\co$ tiene asociada como variedad anal\'itica su divisor cero $[Z]=Z(g)$
o como ya vimos, a la gr\'afica $\Gamma_{f}\subset\Omega\times\co$ como subvariedad anal\'itica de dimensi\'on pura $n$. M\'as a\'un, como sabemos en el contexto
de la variable compleja, toda funci\'on holomorfa en una variedad holomorfa compacta es constante, por ejemplo $\cp^n$ es compacta
y por lo tanto toda funci\'on holomorfa global es constante, sin embargo como sabemos $\cp^n$ tiene muchas subvariedades anal\'iticas
tantas como subconjuntos de funciones polinomiales homogeneas algebraicamente independientes. As\'i mostramos la siguiente tabla con
versiones de resultados cl\'asicos de an\'alisis complejo en el contexto de variedades anal\'iticas 
\\      
\label{table_complex_analysis} 
        \begin{tabular}{| m{5.5cm} | m{5.5cm} |} \hline
                        \begin{center} \vspace*{0.2cm} 
                                \underline{\textbf{An\'alisis Complejo}} 
                        \end{center} & 
                        \begin{center} \vspace*{0.2cm}
                                \underline{\textbf{Conjuntos anal\'iticos en $\con$}}
                        \end{center} \\
                \hline
                \begin{center} 
                        Teorema de Liouville 
                \end{center} & 
                \begin{center}
                        Teorema de Bishop (Teorema \ref{Bishop1})
                \end{center}\\ 
                        \hline $\vert f(z)\vert\leq C\,R^k$ en el conjunto $\{\vert z\vert\leq R\}$ 
                        para todo $R\in\re^{+}$con $f$ entera y $k\in\zah^{+}$, entonces 
                        $f$ es un polinomio. 
                        &
                        \vspace{0.1cm}
                        $Vol_{2k}(X\cap B(R,0))\leq CR^{2k}$ para todo
                        $R\in\re^{+}$, donde $X$ es una subvaredad anal\'itica en $\con$, entonces $X$ es algebraica.\\ 
                        \hline
                        \vspace{0.1cm}
                        \begin{center} 
                        Teorema de extensi\'on de Riemann. 
                \end{center} 
                        & 
                \begin{center} 
                        Generalizaci\'on del teorema de Remmert-Stein (\cite{R-S}) de Bishop. 
                \end{center} \\ 
                        \hline Si $f:(\Omega\setminus E)\subset\co\rightarrow\co$ es una funci\'on holomorfa y $E$ es un subconjunto
                        compacto de capacidad $0$, entonces $f$ es extendible a una funci\'on holomorfa
                        en la regi\'on completa $\Omega$.  
                        &

                        \vspace{0.1cm}
                        Sea $U\subset\con$ un abierta acotado de $\con$ y sea $B\subset U$ un subconjunto cerrado
                        tal que $X\subset U\setminus B$ es una subvariedad de dimensi\'on pura $k$ tal que $B\subset\overline{X}$. 
                        Si $B$ tiene capacidad $0$ relativa a el \'algebra de funciones anal\'iticas en $X$ que son
                        continuas en $\overline{X}$ y si existe una funci\'on $f:U\rightarrow\co^k$ propia en $B$ tal que $f(B)$ 
                        no sea un subconjunto abieto conexo de $\co^k$, entonces $\overline{X}\cap U$ es un subconjunto anal\'iico
                        de $U$ (v\'ease \cite{Bishop}[Theorem 4]).\\ 
                        \hline 
                \begin{center} 
                        Teorema de compacidad de Montel. 
                \end{center} 
                        & 
                \begin{center}
                Teorema de sucesiones de variedades anal\'iticas con volumen uniformemente acotado de Bishop.
                \end{center}\\
                \hline 
                \vspace{0.1cm}
                Sea $\lbrace\Gamma_i\rbrace$ una sucesi\'on de gr\'aficas de funciones holomorfas uniformemente  
                acotadas, $f_i:\Delta\rightarrow\co$ tales que $\Gamma_i\overset{d_H}\longrightarrow\Gamma$ (convergencia de Hausdorff), 
                donde $\Gamma\subset\co^2$ es un subconjunto cerrado y $\Delta$ es el disco unitario en $\co$, 
                entonces $\Gamma$ es la gr\'afica de una funci\'on holomorfa.  
                        & 
                Sea $\lbrace V_i\rbrace$ una sucesi\'on de subvaridades anal\'iticas de $\con$ con volumen 
                uniformemente acotados tales que $V_i\overset{d_H}\longrightarrow V\subset\con$ en el sentido de Hausdorff, entonces 
                $V$ es una suvariedad anal\'itica de $\con$ (\cite{Stolzenberg}[pp. 30]). \\ \hline 
                
\end{tabular} 
\\
Adem\'as de la geometr\'ia compleja y el an\'alisis complejo, otra aplicaci\'on de estos resultados y en particular del teorema
\ref{Bishop2}, es el siguiente resultado sobre foliaciones holomorfas en variedades de K\"ahler con hojas compactas y volumen 
uniformemente acotado. Recordemos que una variedad K\"ahler es una variedad $(M,I)$ compleja, donde $M$ es una variedad 
diferenciable real de dimensi\'on $2d$ e $I$ es una estructura compleja y $h=g-i\omega$ una m\'etrica hermiteana tal que la (1,1)-forma $\omega$ sea cerrada
es decir $d\,\omega=0$. Observamos que $g$ es una m\'etrica riemanniana con la misma forma de volumen que la inducida por $\omega$. Ahora 
una foliaci\'on holomorfa $\mathfrak{F}$ es una distribuci\'on en $H^0(M,TM)$ y geom\'etricamente se puede pensar como una partici\'on 
de $M=\bigcup\mathcal{L}_z$, donde cada $\mathcal{L}_z$ es una subvariedad holomorfa de $(M,I)$. As\'i demostramos lo siguiente:
\begin{teorema}\label{EMS*}
        Sea $M$ una variedad compacta K\"ahler conexa de dimensi\'on compleja $n$, es decir $2n$ real, y $\mathfrak{F}$ una foliaci\'on holomorfa por hojas
        compactas de dimensi\'on real $2d$ donde $d<n$, entonces:
        \begin{enumerate}
                \item[a)] El volumen con respecto a la m\'etrica K\"ahler de las hojas es uniformemente acotado.
                \item[b)] El espacio cosiente $M/\mathfrak{F}$ es un orbifold complejo con singularidades en las hojas de holonomia no trivial.
        \end{enumerate}
\end{teorema}
M\'as a\'un demostramos que la funci\'on volumen $\nu:M/\mathfrak{F}\rightarrow\re^{+}$ definida por el volumen 
\[
        \nu(\mathcal{L}_z):=Vol_{2d}(\mathcal{L}_z)
\]
es discretemente semicontinua inferiormente, es decir que para todo $n\in\zah^{+}$ y $z\in M$ se cumple que existe una vecindad
tal que para todo $\epsilon\in\re^{+}$ 
\[
        \nu(y)>n\nu(z)\hspace{0.2cm}\text{o}\hspace{0.2cm}|\nu(y)-k\nu(z)|<\epsilon\hspace{0.2cm}\text{para alg\'un}\hspace{0.2cm}k\in\{1,\dots,n\}.
\]
\noindent M\'as a\'un los brincos en la continuidad corresponden a las hojas con holonom\'ia no trivial, las cuales son cubiertas
por hojas con holonom\'ia trivial, como todas las hojas son compactas todas, entonces la holonom\'ia es finita y el volumen de las hojas
con holonom\'ia no trivial es una fraci\'on del volumen de las hojas con holonom\'ia trivial. Luego el teorema de generalizado de 
estabilidad de Reeb \cite{Thurston} nos dice c\'omo obtener las cartas coordenadas de $M/\mathfrak{F}$, en el caso de las hojas con holonom\'ia 
trivial, para cada hoja $\mathcal{L}$ existe una vecindad saturada es decir $U=\bigcup_{z\in U}\mathcal{L}_z$ de la hoja tal que $U$ es biholomorfa a $\mathcal{L}\times B$ donde
$B\subset\co^{n-d}$ es una bola, adem\'as cada hoja en $U$ es biholomorfa a $\mathcal{L}\times\{w\}$ con $w\in B$. Esto quiere decir que $M/\mathcal{F}$ tiene una carta coordenada
holomorfa a una bola de dimensi\'on compleja $n-d$ en cada hoja con holonom\'ia trivial. Ahora en el caso de que la holonom\'ia 
no sea trivial, sabemos que el grupo de holonom\'ia es finito y as\'i el teorema generalizado de Reeb \cite{Thurston} nos dice que
igual existe un abierto saturado $U$ cuyo grupo de estructura es el grupo de holonom\'ia el cual es finito, es decir que las hojas
en dicho abierto son cubrientes de nuestra hoja original cuyo grupo de estructura es el grupo de holonom\'ia, por lo tanto
$M/\mathcal{F}$ es un orbifold complejo.

Continuando con esta l\'inea de invaestigaci\'on, notamos que en una variedad de K\"ahler conexa la existencia de una hoja compacta 
con holonom\'ia finita, implica que todas las hojas lo son, esto tambi\'en lo pudimos demostrar utilizando el teorema \ref{Bishop2} a 
diferencia de la prueba en \cite{Pereira}. Todo lo anteriormente mencionado nos pone en el contexto de la conjetura de Beauville:
\begin{conjetura}
        Sea $M$ una varidead compacta K\"ahler tal que exista una descomposici\'on holomorfa de su haz tangente
        \[
        TM=\bigoplus_{i\in I}\mathcal{F}_i\hspace{0.2cm}\text{tal que cada}\hspace{0.2cm}\bigoplus_{i\in J}\mathcal{F}_i\,,\,J\subset I\hspace{0.2cm}\text{es involutivo},
        \]
        entonces el cubriente univeral de $M$ es isomorfo a un producto 
        \[
        \widetilde{X}\cong\prod_{i\in I}U_i\hspace{0.2cm}\text{de tal forma que esto induce}\hspace{0.2cm}T\widetilde{X}\cong\bigoplus_{i\in I}\mathcal{F}_i
        \]
\end{conjetura}
Recientemente, Druel, Pereira, Pym y Touzet demostraron una veri\'on de esta conjetura en un contexto similar pero con el enfoque
particular de variedades de Poisson, v\'ease \cite{DPPT}
\begin{teorema}
        Sup\'ongase que $M$ es una variedad compacta K\"ahler tal que su haz tangente se escinda $TM=\mathfrak{F}\oplus\mathfrak{G}$, donde
        los subhaces $\mathfrak{F}$ y $\mathfrak{G}$ son involutivos. Si $\mathfrak{F}$ tiene una hoja compacta $L$ con holonom\'ia finita,
        entonces $\widetilde{M}$ es biholomorfa a un producto de variedades $N\times P$ cuyas haces tangente son isofomorfos
        a $\mathfrak{F}$ y $\mathfrak{G}$ respectivamente.
\end{teorema}
\noindent Nosotros creemos que es posible demostrar esta proposici\'on por medio de v\'ias distintas a las expuestas en \cite{DPPT} que 
expongan m\'as sobre la estructura anal\'ica del cubriente universal de las hojas, el cual claramente es el mismo para todas estas
y por lo tanto todas las hojas tienen la misma uniformizaci\'on.
Pretendemos primero demostrar el caso en el que las hojas sean compactas, as\'i como hemos visto podemos garantizar que el espacio de
hojas es un orbifold compacto y complejo. Sin embargo quiz\'a sea posible prescindir de esta supocici\'on y solamente suponer que las hojas son
de volumen localmente acotado, bajo esta suposici\'on podemos garantizar que el espacio de hojas es un espacio anal\'itico complejo, adem\'as
la foliaci\'on es localmente una fibraci\'on. Bajo estas suposiciones creemos que es posible definir un biholomorfismo entre los cubrientes
universales de las hojas por medio de levantamiento de curvas y trabajar con el grupoide de holonomia de la foliaci\'on. Es posible que adem\'as
sea posible extraer informaci\'on m\'etrica de las hojas. 

Es claro por lo que hemos expuesto aqu\'i que existe un v\'inculo importante entre la estructura de un espacio anal\'itico y su volumen,
en el caso de las foliaciones podemos extender esta noci\'on a los vol\'umenes de sus hojas adem\'as de que los resultados de Bishop 
nos otorgan un puente entre geometr\'ia y ana\'lisis, por lo que proponemos es estudiar v\'inculos m\'as profundos entre estas
dos \'areas utilizando las herramientas previamente expuestas adem\'as de otros m\'etodos de la geometr\'ia compleja moderna.
Como ya aludimos previamente, las foliaciones holomorfas en variedades de tipo K\"ahler son de particular inter\'es en este aspecto
y por lo tanto es en este contexto que pensamos que la expansi\'on de nuestra investigaci\'on podr\'ia ser m\'as fruct\'ifera en la
busqueda de resultados nuevos. Una de las posibles l\'ineas de investigaci\'on que es enfocarse en relajar las hip\'otesis del 
teorema \ref{EMS*}, lo cual nos deja con las siguientes posibilidades:
\begin{enumerate}
        \item Las hojas no son compactas, s\'olo de volumen finito y localmente acotado en la variedad.
        \item La variedad $M$ no sea compacta.
\end{enumerate}
\noindent Hemos Observado ya que el volumen localmente acotado es sufieciente para garantizar que el espacio de hojas sea anal\'itico 
por lo que queremos explorar otras posibles hip\'otesis para restringir el posible comportamiento de la funci\'on de vol\'umenes
delas hojas.
Otra observaci\'on que por lo menos en el contexto real, es posible encontrar foliacion en variedades compactas real ana\'iticas e
incluso algebraicas, donde todas las hojas de la foliaci\'on son curvas cerradas (c\'irculos) cuyas longitudes no se encuentren 
uniformemente acotados v\'ease \cite{E-V}, algo que es importante destacar de este ejemplo es que la codimensi\'on es lo
suficientemente grande para garantizar que el volumen (longitud) no est\'e uniformemente acotado.


\begin{thebibliography}{3}
\bibitem{A-V} Alexander J.C., Verjovsky A., \textit{First Integrals for Singular Holomorphic Foliations With Leaves of Bounded Volume}
Holomorphic Dynamics. Lecture Notes in Mathematics, vol 1345. Springer, Berlin, Heidelberg.

\bibitem{Beuville1} A. Beauville. (2000) \textit{Complex maifolds with spit tangent bundle}, Complex analysis and algebraic geometry
, de Gruyer, Berlin, 2000.

\bibitem{Bishop} Bishop, E. (1964) \textit{Conditions for the Analyticity  of certain sets}, Michigan Math. J. 11, No. 4, 289--304. 

\bibitem{Chirka} Chirka E. M. (1989) \textit{Complex Analytic Sets}, Kluwer
Academic, Dordrecht,  The Netherlands. 

\bibitem{Chow} Chow, W-L. (1949) \textit{On Compact Complex Analytic Varieties},
American Journal of Mathematics, Vol. 71, No. 4, pp. 893-914.

\bibitem{EMS} Edwards, R., Millet, K., Sullivan, D. (1975) \textit{Foliations
With All Leaves Compact}, Topology Vol. 16, pp. 13-32. Pergamon Press, 1977.

\bibitem{V-A} Epstein, D. B. A., Millet, K. C., Tischler D.
(1977) \textit{Leaves Without Holonomy}, Journal of the London Mathematical
Society.

\bibitem{E-V} Epstein, D. B. A., Voght, E. (1978) \textit{A Counterexample to the Periodic Orbit Conjecture}, 
Annals of Mathematics, Vol. 108, pp. 539-552. 

\bibitem{Epstein1} Epstein, D. B. A.,(1976) \textit{Foliations with all leaves compact}, Annales de l'Institute Fourier, 
tome 26 no. 1, pp. 265-282.

\bibitem{Epstein2} Epstein, D. B. A.,(1977) \textit{Periodic Flows on Three-Manifolds}, Annals of Mathematics, 
Second Series, Vol. 95, No. 1 (Jan., 1972), pp. 66-82.

\bibitem{Thurston} Thurston W, P. (1974) \textit{A Generalization of Reeb Stability Theorem}, Topology Vol. 13.pp. 347-352.
Pergamon Press, Great Britain.
 
\bibitem{R-S} Remmert R., Stein, K. (1953) \textit{Über die wesentlichen
Singulariäten analyscher Mengen}. Math. Annalen, Bd. 126, S. 263--306.

\bibitem{Pereira} Pereira J. V., (2001) \textit{Global Stability for Holomorphic Foliations in Kaehler Maniforlds}. 
Qualitative Theory of Dynamical Systems volume 2, pages 381–384 (2001).

\bibitem{DPPT}Druel S., Pereira J V., Pym B., Touzet F., \textit{A global Weinstein splitting theorem for 
holomorphic Poisson manifolds}, TBP.

\bibitem{G-R} Robert C. (1965) \textit{Analytic Functions of Several
Complex Variables}, Prentice-Hall, Inc, Englewood Cliffs, N.J.

\bibitem{S-M} Scardua, B., Morales C. (2003) \textit{Geometry, Dynamics
and Topology of Foliated Manifolds}. Publicac\~oes Matem\'aticas, Instituto
Nacional de Matem\'atica Pura e Aplicada.

\bibitem{GAGA} Serre, J-P. (1956) \textit{G\'eom\'etrie alg\'ebrique et
g\'eom\'etrie analytique}, Annales de l'institut Fourier, Vol. 6. 

\bibitem{Stolzenberg} Stolzenberg G. (1966) \textit{Volumes, Limits and
Extensions of Analytic Varieties}, Lecture Notes in Mathematics,
Springer-Verlag, Berlin. 

\bibitem{Rudin} Rudin W. (1980) \textit{Function Theory in the Unit Ball of
$\con$}, Springer-Verlag Berlin Heidelberg 2008.

\end{thebibliography}

\end{document}
