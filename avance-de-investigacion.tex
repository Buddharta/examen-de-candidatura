\documentclass[letterpaper]{report}
\usepackage{pst-node}
\usepackage{tikz-cd} 
\usepackage{amsmath}
\usepackage{float}
\usepackage{amsfonts}
\usepackage{amssymb}
\usepackage[spanish,activeacute]{babel}
\usepackage{amscd}
\usepackage{color}
\usepackage{transparent}
\graphicspath{{./figs/}}
\usepackage{makeidx}
\usepackage{afterpage}
\usepackage{array}


\newtheorem{teorema}{Teorema}[chapter]
\newtheorem{prop}[teorema]{Proposici\'on}
\newtheorem{cor}[teorema]{Corolario}
\newtheorem{lema}[teorema]{Lema}
\newtheorem{def.}{Definici\'on}[chapter]
\newtheorem{afir}{Afirmaci\'on}
\newtheorem{conjetura}{Conjetura}

\renewcommand{\figurename}{Figura}
\renewcommand{\chaptername}{\Large  \sc Cap\'{\i}tulo}
\renewcommand{\indexname}{\'{I}ndice anal\'{\i}tico}
\newcommand{\zah}{\ensuremath{ \mathbb Z }}
\newcommand{\nat}{\ensuremath{ \mathbb N }}
\renewcommand{\bibname}{Bibliograf\'{\i}a}
\newcommand{\dem}{{\sc Demostraci\'on. }}
\newcommand{\bg}{\ensuremath{\overline \Gamma}}
\newcommand{\ga}{\ensuremath{\Gamma}}
\newcommand{\fb}{\ensuremath{\overline f}}
\newcommand{\la}{\ensuremath{\lambda}}
\newcommand{\La}{\ensuremath{\Lambda}}
\newcommand{\bt}{\ensuremath{\overline T}}
\newcommand{\li}{\ensuremath{\mathbb{L}}}
\newcommand{\ord}{\ensuremath{\mathbb{O}}}
\newcommand{\bs}{\ensuremath{\mathbb{S}^1}}
\newcommand{\co}{\ensuremath{\mathbb C }}
\newcommand{\con}{\ensuremath{\mathbb{C}^n}}
\newcommand{\cp}{\ensuremath{\mathbb{CP}}}
\newcommand{\rp}{\ensuremath{\mathbb{RP}}}
\newcommand{\re}{\ensuremath{\mathbb R }}
\newcommand{\hc}{\ensuremath{\widehat{\mathbb C} }}
\newcommand{\pslz}{\ensuremath{PSL(2,\mathbb Z) }}
\newcommand{\pslr}{\ensuremath{PSL(2,\mathbb R) }}
\newcommand{\pslc}{\ensuremath{PSL(2,\mathbb C) }}
\newcommand{\qed}{\ensuremath{\hspace*{0em plus 1fill}\blacksquare}}
\newcommand{\hd}{\ensuremath{\mathbb H^2}}

%%\makeindex
%%\renewcommand*{\contentsname}{Temario}
\begin{document}
\begin{titlepage}
\begin{center}

\LARGE\textbf{ UNIVERSIDAD NACIONAL AUT'ONOMA DE M\'EXICO}
\vspace*{0.3cm}

\small PROGRAMA DE MAESTR'IA Y DOCTORADO EN CIENCIAS MATEM'ATICAS Y DE LA ESPECIALIZACI'ON EN ESTAD'ISTICA APLICADA
\vspace*{2cm}

\large\textbf{PRESENTACI\'ON DE PROYECTO DE INVESTIGACI'ON DOCTORAL PARA EL EXAMEN DE CANDIDATURA}
\vspace*{2cm}

\small\textbf{PRESENTA:}
\vspace*{0.2cm}

\small CARLOS EDUARDO MART'INEZ AGUILAR
\vspace*{0.2cm}

\small TUTOR: DR. SANTIAGO ALBERTO VERJOVSKY SOL'A
\vspace*{0.2cm}

\small \textbf{INSTITUTO DE MATEM'ATICAS UNAM}
\vspace*{2.5cm}

\end{center}
\end{titlepage}

%%\tableofcontents

%%\section{Introducci\'on}
\noindent Presento a continuaci\'on el temario y avances de las investigaciones realizadas durante los primeros cinco semestres 
de mi precandidatura a estudiante doctoral como parte del proceso de examen de candidatura a doctorante.
Durante el periodo indicado, tanto mi director de tesis (Dr. Santiago Alberto Verjovsky Sol\'a) como yo nos dedicamos a estudiar 
foliaciones holomorfas $\mathfrak{F}$ cuyas hojas cumplan con ciertas propiedades geom\'etricas que permiten garantizar cierta regularidad en 
la estructura del espacio de hojas de la foliaci\'on. Nos referimos a regularidad en el sentido de que normalmente el 
espacio de hojas es topol\'ogicamente complejo (no Hausdorff), por lo que es necesario restringir el tipo de din\'amicas en las hojas 
de la foliaci\'on para garantizar cierta regularidad topol\'ogica en el espacio de hojas. Un ejemplo de estas propiedades es el caso de foliaciones cuyas hojas tengan 
vol\'umenes acotados en variedades de K\"ahler $(M,h)$. 
La base hist\'orica de estudiar foliaciones con volumen acotado, proviene de trabajos previos hechos por matem\'aticos como
Epstein, Edwards, Millet, Reeb, Sullivan, Haeflinger, J.C. Alexander y el mismo Dr.Verjovsky entre otros, v\'ease \cite{Epstein1}, \cite{Epstein2}, \cite{EMS}, \cite{E-V}, \cite{V-A}. 
Sin embargo, la investigaci\'on que hemos propuesto hasta ahora se encuentra apoyada por el trabajo previo de J.C. Alexander y el 
Dr.Verjovsky \cite{V-A} el cual a su vez se basa en trabajos de Erret Bishop \cite{Bishop} sobre extensiones y l\'imites de Hausdorff 
de sucesiones de variedades anal\'iticas y sigue el esp\'iritu de la presentaci\'on de los resultados de Bishop que se presentan en el 
libro ``Volumes, Limits and Extensions of Analytic Varaieties'' de Gabriel Stolzenberg \cite{Stolzenberg}.

M\'as precisamente, hemos estado investigando maneras de extender el rango de las posibles aplicaciones de estos resultados a
distintas \'areas del an\'alisis complejo y la geometr\'ia compleja, con \'enfasis en la teor\'ia de foliaciones holomorfas. En el trabajo
ya realizado hemos encontrado conexiones entre teoremas conocidos del an\'alisis complejo y la geometr\'ia compleja e incluso hemos
escrito un pequeño art\'iculo de divulgaci\'on con estos hallazgos, el cual por el momento se encuentra en una etapa de correcciones
y posibles expansiones. Uno de los primeros hallazgos de estas conexiones es, por ejemplo, una pueba novedosa del teorema de Chow \cite{Chow}:
\begin{teorema}[Chow]\label{Chow}
        Todo subespacio anal\'itico del espacio proyectivo complejo $\cp^{n}$ es algebraico.
\end{teorema}
\noindent Esto a partir de los resultados de Bishop sobre variedades anal\'iticas de dimensi\'on pura $k$. Aqu\'i entendemos como espacio 
anal\'itico o variedad anal\'itica a un espacio topol\'ogico Hausdorff $X$ paracompacto con una estructura anillada $\mathcal{H}_X$, es decir, una 
gavilla de anillos en $X$ que es localmente isomorfa a un subconjunto anal\'itico de $\con$. M\'as precisamente, un conjunto es anal\'itico 
si para cada $x\in X$ existe una vecindad abierta $V$ y un homeomorfismo $\varphi_V:V\rightarrow Z,$ donde $U\subset\con$ es un abierto y $Z=Z(f_1,\dots,f_k)(U)$ 
representa al subconjunto anal\'itico de los ceros de las funciones holomorfas $f_i:U\rightarrow\co$ y el pullback $\varphi^{*}:\mathcal{H}(U)/\mathcal{I}(Z)\rightarrow \mathcal{H}_X(V)$ 
es un isomorfismo de anillos entre $\mathcal{H}_X(V)$ y el anillo cosiente de
\[
        \mathcal{H}(U)=\{f:U\rightarrow\co\,|\,\text{f es holomorfa}\}\hspace{0.2cm}\text{con}\hspace{0.2cm}\mathcal{I}(Z)=\{f\in\mathcal{H}(U)\,|\,f|_{Z}=0\}.
\]
As\'i un subespacio anal\'itico $Y$ de $X$ es un espacio anillado $(Y,\mathcal{H}_Y)$ con un encaje cerrado $\iota:Y\rightarrow X$ tal que localmente el anillo
$\mathcal{H}_Y$ es isomorfo al anillo cosiente de $\mathcal{H}$ con el ideal de funciones nulas en la imagen de $Y$, donde el isomorfismo se da por medio del 
pullback $\iota^{*}$. Similarmente, la definici\'on de conjunto algebraico se obtiene reemplazando el anillo de funciones anal\'iticas $\mathcal{H}(U)$ con el anillo de 
polinomios en $\con$ denotado por $\co[z_1,\dots,z_n]$.

Siguiendo la presentacion del libro de Stolzenberg \cite{Stolzenberg} el resultado que liga 
los resultados de Bishop con el teorema de Chow es un resultado que estipula lo siguiente:
\begin{teorema}[Bishop]\label{Bishop1}
        Sea $X\subset\con$ una subvariedad anal\'itica de dimensi\'on pura $k$ (compleja), si para toda $R\in\re^+$
        $$Vol_{2k}(X\cap B_R(0))\leq CR^{2k},$$
        donde $B_R(0)$ denota a la bola centrada en el origen de radio $R$ y $Vol_{2k}$ es el volumen de dimensi\'on $2k$. 
        Entonces $X$ es algebraica.
\end{teorema}
\noindent La demostraci\'on que dimos en nuestro art\'iculo es de un caracter muy distinto a la demostraci\'on famosa de
Serre (v\'ease \cite{GAGA}) y nos habla de una conexi\'on interesante entre anal\'isis y geometr\'ia algebraica.
Esta conexi\'on entre an\'alisis y geometr\'ia compleja nos llev\'o a explorar m\'as sobre las posibles conexiones entre estas dos ramas,
adem\'as de otros resultados con el an\'alisis complejo, un ejemplo es el siguiente resultado cl\'asico que demostramos
a partir de los resultados de Bishop 
\begin{teorema}[Montel]\label{Montel}
        Sea $B\subset\con$ la bola unitaria, si $\mathcal{H}(\overline{B})$ es el \'algebra de Banach de funciones holomorfas que son continuas 
        en la fontera de $B$, entonces toda familia de funciones localmente acotadas $F\subset\mathcal{H}(\overline{B})$ es una familia \textit{normal}.
\end{teorema}
Recordamos que una familia de funciones $F$ es \textit{normal} si y s\'olo si su cerradura es secuencialmente compacta. Si vemos este 
teorema desde la perpectiva de los teoremas de Bishop, entonces podemos considerar que el siguiente teorema es una generalizaci\'on en
un contexto m\'as general.
\begin{teorema}\label{Bishop2}
        Sea $\{ V_n \}_{n\in\nat}$ una sucesi\'on de variedades anal\'iticas de dimensi\'on pura $k$ en $\con$ con volumen uniformemente
        acotado por una constante $C\in\re^{+}$. Si $V_n\rightarrow V$ converge a un cerrado $V\subset\con$ en el sentido de Hausdorff, entonces 
        $V$ es una variedad anal\'itica. (v\'ease \cite{Stolzenberg}[pp. 30])
\end{teorema}
Convergencia de una sucesi\'on de conjuntos $S_n\rightarrow S$ en el sentido de Hausdorff para conjuntos cerrados de un espacio m\'etrico $(X,d)$
sucede cuando $S_n\cap K\rightarrow S\cap K$ en el sentido de la m\'etrica de Hausdorff $d_H$ para todo $K\subset X$ compacto, donde
\[
        d_H(K_1,K_2):= \max_{x\in K_1}\left\{d(x,K_2)\right\}+\max_{y\in K_2}\left\{d(y,K_1)\right\}. 
\]
Nosostros demostramos el resultado de Montel por medio del teorema \ref{Bishop2} utilizando las gr\'aficas de las
funciones 
\[
        \Gamma_{f}=\{(z,w)\in B\times\co\,|\,w=f(z)\}=Z(w-f(z)),
\]
\noindent donde para una funci\'on holomorfa $g:\Omega\rightarrow\co$ denotamos a su conjunto nulo como $Z(g)=\{z\in\Omega\,|\,g(z)=0\}$. Ahora aqu\'i 
pensamos a $\Gamma_{f}$ como variedades anal\'iticas de dimensi\'on pura $n$ en un abierto de $\co^{n+1}$, en el art\'iculo demostramos 
que para toda sucesi\'on $\{f_n\}_{n\in\nat}\subset F$ de funciones en una familia localmente acotada, tiene una subsucesi\'on 
cuyas gr\'aficas convergen en el sentido de la m\'etrica de Hausdorff, adem\'as demostramos que su conjunto l\'imite 
es a su vez la gr\'afica de una funci\'on holomorfa en $B$.

Como hemos visto, los resultados de Bishop se pueden pensar como generalizaciones de resutados cl\'asicos del an\'alisis complejo,
pero vistos dentro de la teor\'ia de variedades anal\'iticas, las cuales se pueden pensar claramente como una generalizaci\'on, 
ya que cada funci\'on holomorfa $f:\Omega\subset\con\rightarrow\co$ tiene asociada como variedad anal\'itica su divisor cero $[Z]=Z(g)$
o como ya vimos a su gr\'afica $\Gamma_{f}\subset\Omega\times\co$ como subvariedad anal\'itica de dimensi\'on pura $n$. M\'as a\'un, como sabemos, en el contexto
de la variable compleja, toda funci\'on holomorfa en una variedad holomorfa compacta es constante, por ejemplo $\cp^n$ es compacta,
y por lo tanto toda funci\'on holomorfa global es constante. Sin embargo, como sabemos $\cp^n$ tiene muchas subvariedades anal\'iticas, tantas
tantas como subconjuntos de funciones polinomiales homog\'eneas algebraicamente independientes. As\'i, mostramos la siguiente tabla con
versiones de resultados cl\'asicos de an\'alisis complejo en el contexto de variedades anal\'iticas 
\\      
\label{table_complex_analysis} 
        \begin{tabular}{| m{5.5cm} | m{5.5cm} |} \hline
                        \begin{center} \vspace*{0.2cm} 
                                \underline{\textbf{An\'alisis Complejo}} 
                        \end{center} & 
                        \begin{center} \vspace*{0.2cm}
                                \underline{\textbf{Conjuntos anal\'iticos en $\con$}}
                        \end{center} \\
                \hline
                \begin{center} 
                        Teorema de Liouville 
                \end{center} & 
                \begin{center}
                        Teorema de Bishop (Teorema \ref{Bishop1})
                \end{center}\\ 
                        \hline $\vert f(z)\vert\leq C\,R^k$ en el conjunto $\{\vert z\vert\leq R\}$ 
                        para todo $R\in\re^{+}$con $f$ entera y $k\in\zah^{+}$, entonces 
                        $f$ es un polinomio. 
                        &
                        \vspace{0.1cm}
                        $Vol_{2k}(X\cap B(R,0))\leq CR^{2k}$ para todo
                        $R\in\re^{+}$, donde $X$ es una subvariedad anal\'itica en $\con$, entonces $X$ es algebraica.\\ 
                        \hline
                        \vspace{0.1cm}
                        \begin{center} 
                        Teorema de extensi\'on de Riemann. 
                \end{center} 
                        & 
                \begin{center} 
                        Generalizaci\'on del teorema de Remmert-Stein (Original: \cite{R-S}) de Bishop \cite{Bishop}, \cite{Stolzenberg}[pp. 34]. 
                \end{center} \\ 
                        \hline Si $f:(\Omega\setminus E)\subset\co\rightarrow\co$ es una funci\'on holomorfa y $E$ es un subconjunto
                        compacto de capacidad $0$, entonces $f$ es extendible a una funci\'on holomorfa
                        en la regi\'on completa $\Omega$.  
                        &

                        \vspace{0.1cm}
                        Sea $U\subset\con$ un abierto acotado de $\con$ y sea $B\subset U$ un subconjunto cerrado
                        tal que $X\subset U\setminus B$ es una subvariedad de dimensi\'on pura $k$ tal que $B\subset\overline{X}$. 
                        Si $B$ tiene capacidad $0$ relativa al \'algebra de funciones anal\'iticas en $X$ que son
                        continuas en $\overline{X}$ y si existe una funci\'on $f:U\rightarrow\co^k$ propia en $B$ tal que $f(B)$ 
                        no sea un subconjunto abieto conexo de $\co^k$, entonces $\overline{X}\cap U$ es un subconjunto anal\'itico
                        de $U$ (v\'ease \cite{Bishop}[Theorem 4]).\\ 
                        \hline 
                \begin{center} 
                        Teorema de compacidad de Montel. 
                \end{center} 
                        & 
                \begin{center}
                Teorema de sucesiones de variedades anal\'iticas con volumen uniformemente acotado de Bishop.
                \end{center}\\
                \hline 
                \vspace{0.1cm}
                Sea $\lbrace\Gamma_i\rbrace$ una sucesi\'on de gr\'aficas de funciones holomorfas uniformemente  
                acotadas, $f_i:\Delta\rightarrow\co$ tales que $\Gamma_i\overset{d_H}\longrightarrow\Gamma$ (convergencia de Hausdorff), 
                donde $\Gamma\subset\co^2$ es un subconjunto cerrado y $\Delta$ es el disco unitario en $\co$, 
                entonces $\Gamma$ es la gr\'afica de una funci\'on holomorfa.  
                        & 
                Sea $\lbrace V_i\rbrace$ una sucesi\'on de subvariedades anal\'iticas de $\con$ con volumen 
                uniformemente acotados tales que $V_i\overset{d_H}\longrightarrow V\subset\con$ en el sentido de Hausdorff, entonces 
                $V$ es una subvariedad anal\'itica de $\con$ (\cite{Stolzenberg}[pp. 30]). \\ \hline 
                
\end{tabular} 
\\
Adem\'as de la geometr\'ia compleja y el an\'alisis complejo, otra aplicaci\'on de estos resultados y en particular del teorema
\ref{Bishop2}, es el siguiente resultado sobre foliaciones holomorfas en variedades de K\"ahler con hojas compactas y volumen 
uniformemente acotado. Recordemos que una variedad K\"ahler es una variedad $(M,I)$ compleja, donde $M$ es una variedad 
diferenciable real de dimensi\'on $2d$ e $I$ es una estructura compleja y $h=g-i\omega$ una m\'etrica hermiteana tal que la (1,1)-forma $\omega$ sea cerrada,
es decir, $d\,\omega=0$. Observamos que $g$ es una m\'etrica riemanniana con la misma forma de volumen que la inducida por $\omega$. Ahora, 
una foliaci\'on holomorfa $\mathfrak{F}$ es una distribuci\'on integrable en $H^0(M,TM)$ (m\'odulo cerrado bajo el corchete de Lie con dimensi\'on constante) 
y geom\'etricamente se puede pensar como una partici\'on de $M=\bigcup\mathcal{L}_z$, donde cada $\mathcal{L}_z$ es una subvariedad 
holomorfa de $(M,I)$. As\'i demostramos lo siguiente:
\begin{teorema}\label{EMS*}
        Sea $M$ una variedad compacta K\"ahler conexa de dimensi\'on compleja $n$, es decir $2n$ real, y $\mathfrak{F}$ una foliaci\'on holomorfa por hojas
        compactas de dimensi\'on real $2d$ donde $d<n$, entonces:
        \begin{enumerate}
                \item[a)] El volumen con respecto a la m\'etrica K\"ahler de las hojas es uniformemente acotado.
                \item[b)] El espacio cociente $M/\mathfrak{F}$ es un orbifold complejo con singularidades en las hojas de holonomia no trivial.
        \end{enumerate}
\end{teorema}
Aunado a esto, demostramos que la funci\'on volumen $\nu:M/\mathfrak{F}\rightarrow\re^{+}$ definida por el volumen 
\[
        \nu(\mathcal{L}_z):=Vol_{2d}(\mathcal{L}_z)
\] 
es discratemente semicontinua inferiormente, es decir, que para todo $n\in\zah^{+}$ y $z\in M$ se cumple que existe una vecindad
tal que para todo $\epsilon\in\re^{+}$ 
\[
        \nu(y)>n\nu(z)\hspace{0.2cm}\text{o}\hspace{0.2cm}|\nu(y)-k\nu(z)|<\epsilon\hspace{0.2cm}\text{para alg\'un}\hspace{0.2cm}k\in\{1,\dots,n\}.
\]
\noindent M\'as a\'un, los brincos en la continuidad corresponden a las hojas con holonom\'ia no trivial, las cuales son cubiertas
por hojas con holonom\'ia trivial. Como todas las hojas son compactas, la holonom\'ia es finita y el volumen de las hojas
con holonom\'ia no trivial es una fracci\'on del volumen de las hojas con holonom\'ia trivial por estabilidad de Reeb \cite{Thurston}. 
Luego, el teorema generalizado de estabilidad de Reeb \cite{Thurston} nos dice c\'omo obtener las cartas coordenadas de $M/\mathfrak{F}$. 
En el caso de las hojas con holonom\'ia trivial, para cada hoja $\mathcal{L}$ existe una vecindad saturada, es decir, $U=\bigcup_{z\in U}\mathcal{L}_z$ de la hoja, 
tal que $U$ es biholomorfa a $\mathcal{L}\times B$ donde $B\subset\co^{n-d}$ es una bola. Adem\'as, cada hoja en $U$ es biholomorfa a $\mathcal{L}\times\{w\}$ con $w\in B$. 
Esto quiere decir que $M/\mathcal{F}$ tiene una carta coordenada holomorfa a una bola de dimensi\'on compleja $n-d$ en cada hoja 
con holonom\'ia trivial. Ahora, en el caso de que la holonom\'ia no sea trivial, sabemos que el grupo de holonom\'ia es 
finito y as\'i el teorema generalizado de Reeb \cite{Thurston} nos dice que igual existe un abierto saturado $U$ 
cuyo grupo de estructura es el grupo de holonom\'ia el cual es finito, es decir, que las hojas en dicho abierto son 
cubrientes de nuestra hoja original cuyo grupo de estructura es el grupo de holonom\'ia, por lo tanto $M/\mathcal{F}$ es un orbifold complejo.
Esto se puede contrastar con la prueba original (v\'ease \cite{EMS}) la cual es m\'as general, pero al mismo tiempo es m\'as 
complicada y ajena a las peculiaridades de la geometr\'ia k\"ahleriana. 

Continuando con esta l\'inea de investigaci\'on, notamos que en una variedad de K\"ahler conexa la existencia de una hoja compacta 
con holonom\'ia finita implica que todas las hojas lo son, esto tambi\'en lo pudimos demostrar utilizando el teorema \ref{Bishop2} a 
diferencia de la prueba en \cite{Pereira}. Todo lo anteriormente mencionado nos pone en el contexto de la conjetura de Beauville (2000):
\begin{conjetura}\label{Beauville}
        Sea $M$ una varidead compacta K\"ahler tal que exista una descomposici\'on holomorfa de su haz tangente
        \[
        TM=\bigoplus_{i\in I}\mathcal{F}_i\hspace{0.2cm}\text{tal que cada}\hspace{0.2cm}\bigoplus_{i\in J}\mathcal{F}_i\,,\,J\subset I\hspace{0.2cm}\text{es involutivo},
        \]
        entonces el cubriente univeral de $M$ es isomorfo a un producto 
        \[
        \widetilde{X}\cong\prod_{i\in I}U_i\hspace{0.2cm}\text{de tal forma que esto induce}\hspace{0.2cm}T\widetilde{X}\cong\bigoplus_{i\in I}\mathcal{F}_i
        \]
\end{conjetura}
Recientemente, Druel, Pereira, Pym y Touzet demostraron una versi\'on de esta conjetura en el contexto que expusimos previamente,
pero con el enfoque particular de variedades de Poisson, v\'ease \cite{DPPT}.
\begin{teorema}
        Sup\'ongase que $M$ es una variedad compacta K\"ahler tal que su haz tangente se escinda $TM=\mathfrak{F}\oplus\mathfrak{G}$, donde
        los subhaces $\mathfrak{F}$ y $\mathfrak{G}$ son involutivos. Si $\mathfrak{F}$ tiene una hoja compacta $L$ con holonom\'ia finita,
        entonces $\widetilde{M}$ es biholomorfa a un producto de variedades $N\times P$ cuyas haces tangentes son isofomorfos
        a $\mathfrak{F}$ y $\mathfrak{G}$ respectivamente.
\end{teorema}
\noindent Nosotros creemos que es posible demostrar esta proposici\'on por medio de v\'ias distintas a las expuestas en \cite{DPPT}, que 
expongan m\'as sobre la estructura anal\'itica del cubriente universal de las hojas, el cual claramente es el mismo para todas \'estas.
Pretendemos primero demostrar el caso en el que las hojas sean compactas, as\'i como hemos visto previamente, podemos garantizar que el espacio de
hojas es un orbifold compacto y complejo.
\\
                                \textcolor{red}{\centerline{Problema a resolver, hip\'otesis:}} 
Es posible prescindir de la suposici\'on de hojas compactas suponiendo solamente que las hojasson de volumen localmente acotado. 
Creemos que bajo esta suposici\'on podemos garantizar que el espacio de hojas es un espacio anal\'itico complejo, adem\'as la 
foliaci\'on es localmente una fibraci\'on. Bajo estas suposiciones, creemos que es posible definir 
un biholomorfismo entre los cubrientes universales de las hojas por medio de levantamientos de curvas y trabajar con el grupoide 
de holonomia de la foliaci\'on similar a lo relizado en \cite{DPPT}. Adem\'as creemos que es posible extraer informaci\'on m\'etrica 
de las hojas (o su cubriente) si se levantan geod\'esicas en lugar de curvas arbitrarias, donde  simplemente utilizamos la 
m\'etrica riemannina en $M/\mathfrak{F}$ heredada de la m\'etrica K\"ahler original. 

Es importante observar que es posible encontrar foliaciones en variedades compactas real anal\'iticas e incluso algebraicas, donde 
todas las hojas de la foliaci\'on son curvas cerradas (c\'irculos) cuyas longitudes no se encuentren uniformemente acotadas 
v\'ease \cite{E-V}. Algo que es importante destacar de este ejemplo es que la codimensi\'on es lo suficientemente grande 
para garantizar que el volumen (longitud) no se encuentre uniformemente acotado y tambi\'en sucede que la funci\'on de longitud de las hojas 
no es localmente acotada.

Es claro por lo que hemos expuesto aqu\'i que existe un v\'inculo importante entre la estructura de un espacio anal\'itico y su volumen.
en el caso de las foliaciones, podemos extender esta noci\'on a los vol\'umenes de sus hojas, adem\'as de que los resultados de Bishop 
nos otorgan un puente entre geometr\'ia y an\'alisis, por lo que proponemos  estudiar v\'inculos m\'as profundos entre estas
dos \'areas utilizando las herramientas previamente expuestas adem\'as de otros m\'etodos de la geometr\'ia compleja moderna.
Como ya aludimos previamente, las foliaciones holomorfas en variedades de tipo K\"ahler son de particular inter\'es en este aspecto
y por lo tanto es en este contexto que pensamos que la expansi\'on de nuestra investigaci\'on podr\'ia ser m\'as fruct\'ifera en la
b\'usqueda de resultados nuevos. 

Hemos observado ya que el volumen localmente acotado es suficiente para garantizar que el espacio de hojas sea anal\'itico, 
por lo que queremos explorar otras posibles hip\'otesis para restringir el posible comportamiento de la funci\'on de vol\'umenes
de las hojas. Una pregunta natural es la siguiente: Si suponemos que el espacio $M$ o el espacio de las hojas $M/\mathfrak{F}$ tienen 
estructura adicional (Poisson, Fano, Stein, Proyectiva) ¿Qu\'e tipo de comportamiento pueden tener las funciones de volumen de
una variedad? Inspirados por \cite{DPPT}, creemos que existen relaciones interesantes entre dicha funci\'on de volumen y la estructura 
de una variedad Fano-Poisson.

Recordamos que una variedad compleja $M$ es de Poisson si existe una operaci\'on bilineal en el anillo de g\'ermenes de 
funciones holomorfas en $M$, la cual se le conoce por el nombre de \textit{corchete de Poisson}. Denotaremos por $\mathcal{O}_M:=\mathcal{H}_M/\sim$, donde $f_1\sim f_2$ si 
$f1=f2|_U$, al anillo de g\'ermenes de funciones holomorfas. As\'i, un corchete de Poisson es una funci\'on bilineal
\[
\{\cdot,\cdot\}:\mathcal{O}_M\times\mathcal{O}_M\rightarrow\mathcal{O}_M,
\]
que cumple las siguientes propiedades
\begin{enumerate}
\item $\{f,g\} = -\{g,f\}$
\item $\{f,gh\}=\{f,g\}h + g\{f,h\}$
\item $\{f,\{g,h\}\}+\{g,\{h,f\}\} + \{h,\{ f,g\}\}=0$.
\end{enumerate}
\noindent Observamos que un germen de una funci\'on fija $H\in\mathcal{O}_M$ define un campo vectorial definido por $\xi_H(\cdot)=\{H,\cdot\}$,
a \'este campo vectorial le llamamos el campo \textit{hamiltoniano} definido por $H$. Utilizando la notaci\'on de campos multivectoriales, 
tenemos que una definici\'on alternativa del corchete de Poisson es la siguiente: denotamos a los campos vectoriales holomorfos como $\mathcal{T}M=H^0(M,TM)$,
entonces el espacio de p-vectores se define por 
\[
        \Lambda^{p}(\mathcal{T}M):=\{\mathcal{O}_M\times\dots\times\mathcal{O}_M\rightarrow\mathcal{O}_M\,\vert\,\text{sesquilineal}\}.
\]
Entonces, claramente un corchete de Poisson es un campo bivectorial, el cual podemos definir por medio del emparejamiento $\langle\cdot,\cdot\rangle$,
entre los campos p-vectoriales y el espacio de p-formas diferenciales holomorfas $\Omega^{p}(M)$ por medio de
\[
        \pi\in H^0(M,\Lambda^2(\mathcal{T}M))\,\text{ entonces }\,\{f,g\}=\langle \pi,df\wedge dg\rangle.
\]
Por lo tanto, generalizando esto tenemos el siguiente mapeo definido por un corchete de Poisson $\pi$
\[
        \pi^{\#}:\Omega^1_M\rightarrow\mathcal{T}M,\hspace{0.2cm}\pi^{\#}(\alpha):=\iota_{\alpha}(\pi):=\langle\pi,\alpha\wedge\cdot\rangle.
\]
\noindent Con esto, el rango de $\pi$ en un punto $p\in M$ se define como el entero positivo $r\in \zah^{+}$ que define la dimensi\'on del
espacio m\'aximo, donde $\pi^{\#}_p$ es no degenerada, es decir, es la dimensi\'on del espacio m\'as grande, tal que $\pi^{\#}$ es una
biyecci\'on. Si $\pi^{\#}$ es no degenerada, su rango es $2n=\dim(M)$, entonces $\pi^{-1}$, la inversa de $\pi^{\#}$, define una forma simpl\'ectica
en $M$. El teorema de escici\'on de Weinstein define una foliaci\'on natural en una variedad $(M,\pi)$ de Poisson, pero primero 
recordamos que una funci\'on holomorfa entre dos variedades de Poisson $(M_1,\{\cdot,\cdot\}_1)$ y $(M_2,\{\cdot,\cdot\}_2)$ es un morfismo de Poisson 
$\phi:M_1\rightarrow M_2$ si $\{f,g\}_1\circ\phi=\{f\circ\phi,g\circ\phi\}_2$
\begin{teorema}
        Sea $(M,\pi)$ una variedad holomorfa de Poisson. Supongamos que $\pi$ es de rango $1\geq 2r\leq 2n=\dim(M)$ en $x\in M$, entonces existe 
        una vecindad $U$ de $x$ tal que $U$ es isomorfa en el sentido de Poisson a un producto $S\times P$ de tal forma que $S$ es
        simpl\'ectica con coordenadas $(p_i,q_i)_{i=1}^r$ y $(P,\tilde{\pi})$ es una variedad de Poisson de rango cero en $x$
        con coordenadas $z=(z_j)_{j=1}^{2n-2r}$
        \[
                \pi=\sum_{i=1}^r \partial{p_i}\wedge\partial{q_i}+\sum_{1\leq j\leq k\leq 2n-2r} f^{jk}(z)\partial{z_j}\wedge\partial{z_k}.
        \]
\end{teorema} 
\noindent Observamos del teorema anterior que $f^{jk}(x)=0$. Este teorema define claramente una foliaci\'on natural en $(M,\pi)$ de hojas
simpl\'ecticas. Sin embargo, no todas son de las misma dimensi\'on, por lo que es necesario notar que la definici\'on de foliaci\'on
se puede expadir a este contexto m\'as general, es decir, una foliaci\'on es simplemente un $\mathcal{O}_M$-subm\'odulo de $\mathcal{T}M$ involutivo.
Ahora, esto define una filtraci\'on $X_0\subset X_2\subset X_4\subset\dots\subset M$, donde $X_{2k}=\{x\in M\,|\,rang(\pi_x)\leq 2k\}$,
si denotamos a las hojas simpl\'ecticas de $(M,\pi)$ como $\mathcal{L}$, entonces tambi\'en podemos pensar a $X_{2k}$ como 
$$
X_{2k}=\bigcup_{\dim(\mathcal{L})\leq 2k}\mathcal{L}.
$$
Con esto establecido, presentaremos una conjetura establecida en 1993 por Bondal para variedades Fano-Poisson que esperamos poder 
iluminar con nueva informaci\'on utilizando la funci\'on volumen. Recordamos que una variedad $M$ es Fano si cumple lo siguiente (vease \cite{S-Yau} y \cite{ZB}):
\begin{itemize}
        \item $M$ admite una m\'etrica de K\"ahler-Einstein, es decir, si definimos la m\'etrica por medio de su forma simpl\'ectica 
        K\"ahler $\omega$, entonces
                $$Ric_{\omega}=\lambda\omega\hspace{0.2cm}\lambda\in\re,$$
        donde en coordendas podemos calcular la curvatura de Ricci para la m\'etrica 
        \hbox{$h=g-i\omega=\sum h_{i\overline{j}}\,dz_i\otimes d\overline{z}_j$} como
        $$Ric_{\omega}=\frac{i}{2\pi}\sum_{ij}R_{i\overline{j}}\,dz_i\wedge d\overline{z}_j=\frac{-i}{2\pi}\partial\overline{\partial}\log(\det(h_{k\overline{l}})).$$
        \item La clase de cohomolog\'ia definida por la curvatura de Ricci (primera clase de Chern) es positiva
        $$
        [Ric_{\omega}]=c_1(M)>0.
        $$
\end{itemize}
\noindent Mencionamos que normalmente en la literatura se define una variedad Fano como una variedad algebraica $X$ completa 
en el sentido de que toda proyecci\'on $X\times Y\rightarrow Y$ es cerrada para toda variedad algebraica $Y$, tal que su divisor/haz antican\'onico
$K^{*}_{X}$ es amplio, por lo tanto toda variedad Fano es proyectiva. La disparidad entre la definici\'on que dimos y la utilizada 
ampliamente en la literatura proviene de la demostraci\'on dada por Yau de la conjeturla de Calabi (v\'ease \cite{S-Yau}).
\begin{conjetura}\label{Bondal}
         Sea $M$ una variedad de Fano-Poisson, bajo la notaci\'on previamente establecida, $X_{2k}$ es la uni\'on de las hojas 
         simpl\'ecticas de dimensi\'on $2k$, entonces $X_{2k}$ tiene una componente de dimensi\'on mayor a $2k$.
\end{conjetura}
\noindent No sabemos si es cierta o falsa o si es posible demostrar con lo que vamos a proponer, pero creemos que como una 
investigaci\'on entre la relaci\'on entre la geometr\'ia complejo-diferencial y la geometr\'ia algebraica lo que proponemos es interesante. 
Esta conexi\'on creemos que es relevante para acercamiento a la geometr\'ia que nosotros proponemos, esto est\'a inspirado por trabajos 
previos de Yau \cite{S-Yau} y sobre todo Donaldson y Sun \cite{D-SS}. En particular, Donaldson y Sun muestran resultados de naturaleza muy similar 
a los resultados de Bishop (teorema \ref{Bishop2}) en el contexto generalizado de l\'imites de Gromov-Hausdorff en variedades de 
K\"ahler, con particular aplicaci\'on a las variedades de Fano.
\\  

                                                \textcolor{red}{\centerline{Proponemos:}}
\begin{itemize}
        \item Estudiar el crecimiento del volumen de los conjuntos $X_{2k}\cap B_{R}(x)$ donde $B_R(x)$ es la bola m\'etrica de radio $R$
        centrada en un punto $x\in X_{2k}$ cuando $R\rightarrow\infty$. Si el conjunto $X_{2k}$ es de dimensi\'on mayor a $2k$, 
        esperamos un comportamiento de crecimiento de volumen mayor a $\mathcal{O}(R^{2k})$.
        
        \item Buscar casos en los que existan descripciones de los conjuntos $X_{2k}$ que nos permitan describir dichos conjuntos
        como l\'imites de Gromov-Hausdorff y hacer uso de las cotas dadas en \cite{D-SS}, especialmente pensando en el caso
        de hojas compactas.

        \item Hacer un contraste entre los aspectos geom\'etricos encontrados y las demostraciones a la conjetura \ref{Bondal} en los
        casos ya demostrados como en \cite{Gua-Pym}.
\end{itemize}

\vspace{1cm}

Por \'ultimo en t\'erminos de comportamiento local de las foliaciones, estudiamos en el caso de dimensi\'on dos foliaciones en 
la bola unitaria $B\subset\co^n$ uniformizadas por el disco $\Delta:=\{|z|< 1\}\subset\co$ buscando su comportamiento l\'imite cuando uno piensa
al disco como una variedad K\"ahler completa. Para esto, pensamos al disco encajado en el espectro del \'algebra de Banach 
$\mathcal{H}^{\infty}=\mathcal{L}(\Delta)^{\infty}\cap\mathcal(\Delta)$, as\'i \hbox{$M^{\infty}:=\{\kappa:\mathcal{H}\rightarrow\co\,|\,\kappa(fg)=\kappa(f)\kappa(g),\,\kappa(f+g)=\kappa(f)+\kappa(g),\,\kappa(1)=1\}$}
y por lo tanto $\Delta$ se encaja en $M^{\infty}$, $\Delta\rightarrow M^{\infty}$ por medio de la funcion evaluaci\'on $ev(z)[f]=f(z)$. 
El conjunto $M^{\infty}$ es un espacio m\'etrico compacto y es un modelo de compactaci\'on para $\Delta$, adem\'as, el conjunto $M^{\infty}\setminus\Delta$ conocido como 
la \textit{corona} (v\'ease \cite{SCHARK}) tiene muchas propiedades interesantes.
Es en este conjunto en el que esperamos modelar los l\'imites de las hojas, donde por l\'imites nos referimos al conjunto al que se 
acercan las geod\'esicas de las hojas en $B$. Definimos el l\'imite como la imagen $\overline{\varphi}(M^{\infty}\setminus\Delta)$, donde $\varphi$ es la uniformizaci\'on
de la hoja y $\overline{\varphi}$ es su extensi\'on a $M^{\infty}$. El contexto en el que nos comenzamos a cuestionar estos aspectos fue
por medio de las observaciones siguientes:
\begin{enumerate}
        \item Toda foliaci\'on en la bola $B\subset\co^2$ con una singularidad en el origen tiene una hoja que pasa por el origen
        por el teorema de Camacho-Sad. As\'i, en $B\setminus\{0\}$ las hojas son incompletas y por lo tanto existe una hoja
        y una geod\'esica cuyo l\'imite es $0$.
        \item Sorprendentemente existen foliaciones no singulares por discos en $B$ donde las hojas son completas, por lo que 
        encontrar los l\'imites de las geod\'esicas en la bola $\overline{B}$ es interesante en esos casos, v\'ease \cite{A-F}.
\end{enumerate}
                                        \textcolor{red}{\centerline{Especulamos:}}
Si $\mathfrak{F}$ es una foliaci\'on con una singularidad aislada en el origen de $B$ cuyas hojas se uniformizan por el disco $\Delta$,
entonces existe una hoja incompleta y una geod\'esica en \'esta cuyo l\'imite es $0$.  

\begin{thebibliography}{3}

\bibitem{A-F} Alarc\'on, A., Forstneric (2019) \textit{A foliation of the ball by complete holomorphic discs}, Mathematische Zeitschrift
pp. 169-174, Springer-Verlag.

\bibitem{A-V} Alexander J.C., Verjovsky A., \textit{First Integrals for Singular Holomorphic Foliations With Leaves of Bounded Volume}
Holomorphic Dynamics. Lecture Notes in Mathematics, vol 1345. Springer, Berlin, Heidelberg.

\bibitem{Beuville1} A. Beauville. (2000) \textit{Complex manifolds with split tangent bundle}, Complex analysis and algebraic geometry
, de Gruyer, Berlin, 2000.

\bibitem{ZB} Blocki, Z., (2011) \textit{The Complex Monge-Amp\`ere Equation in K\"ahler Geometry} Lecture Notes in Mathematics 2075
paart of Pluripotential Theory, Springer-Verlag (2011), pp. 95-143.

\bibitem{Bishop} Bishop, E. (1964) \textit{Conditions for the Analyticity  of certain sets}, Michigan Math. J. 11, No. 4, 289--304. 

\bibitem{Bondal} Bondal, A I,. (1993) \textit{Non-commutative deformations and Poisson brackets on projetive spaces}, 
Max-Planck-Institut f\"ur Matematik, Germany.

\bibitem{Chirka} Chirka E. M. (1989) \textit{Complex Analytic Sets}, Kluwer
Academic, Dordrecht,  The Netherlands. 

\bibitem{Chow} Chow, W-L. (1949) \textit{On Compact Complex Analytic Varieties},
American Journal of Mathematics, Vol. 71, No. 4, pp. 893-914.

\bibitem{D-SS} Donaldson, S., Sun, S. (2014) \textit{Gomov-Hausdorff Limits of K\"ahler Manifolds and Algebraic Geometry I}
 Acta Math 213, 63–106, Springer-Verlag.

\bibitem{DPPT}Druel S., Pereira J V., Pym B., Touzet F. \textit{A global Weinstein splitting theorem for 
holomorphic Poisson manifolds}, TBP.

\bibitem{EMS} Edwards, R., Millet, K., Sullivan, D. (1975) \textit{Foliations
With All Leaves Compact}, Topology Vol. 16, pp. 13-32. Pergamon Press, 1977.

\bibitem{V-A} Epstein, D. B. A., Millet, K. C., Tischler D.
(1977) \textit{Leaves Without Holonomy}, Journal of the London Mathematical
Society.

\bibitem{E-V} Epstein, D. B. A., Voght, E. (1978) \textit{A Counterexample to the Periodic Orbit Conjecture}, 
Annals of Mathematics, Vol. 108, pp. 539-552. 

\bibitem{Epstein1} Epstein, D. B. A. (1976) \textit{Foliations with all leaves compact}, Annales de l'Institute Fourier, 
tome 26 no. 1, pp. 265-282.

\bibitem{Epstein2} Epstein, D. B. A. (1977) \textit{Periodic Flows on Three-Manifolds}, Annals of Mathematics, 
Second Series, Vol. 95, No. 1 (Jan., 1972), pp. 66-82.

\bibitem{Gua-Pym} Gualtieri, M., Pym, B. (2012) \textit{Poisson modules and degeneracy loci},
Proceedings of the London Mathematical Society 107(3).

\bibitem{R-S} Remmert R., Stein, K. (1953) \textit{Über die wesentlichen
Singulariäten analyscher Mengen}. Math. Annalen, Bd. 126, S. 263--306.

\bibitem{G-R} Robert C. (1965) \textit{Analytic Functions of Several
Complex Variables}, Prentice-Hall, Inc, Englewood Cliffs, N.J.

\bibitem{Rudin} Rudin W. (1980) \textit{Function Theory in the Unit Ball of
$\con$}, Springer-Verlag Berlin Heidelberg 2008.

\bibitem{GAGA} Serre, J-P. (1956) \textit{G\'eom\'etrie alg\'ebrique et
g\'eom\'etrie analytique}, Annales de l'institut Fourier, Vol. 6. 

\bibitem{SCHARK} SCHARK, I. J. \textit{Maximal Ideals in an Algebra of Bounded Analytic Functions}, The Institute for Advanced Study
Princeton, New Jersey.

\bibitem{Stolzenberg} Stolzenberg G. (1966) \textit{Volumes, Limits and
Extensions of Analytic Varieties}, Lecture Notes in Mathematics,
Springer-Verlag, Berlin. 

\bibitem{Thurston} Thurston W, P. (1974) \textit{A Generalization of Reeb Stability Theorem}, Topology Vol. 13.pp. 347-352.
Pergamon Press, Great Britain.

\bibitem{Pereira} Pereira J. V., (2001) \textit{Global Stability for Holomorphic Foliations in Kaehler Maniforlds}. 
Qualitative Theory of Dynamical Systems volume 2, pages 381–384 (2001).

\bibitem{S-Yau} Yau, S-T., (1978) \textit{On The Ricci Cirvature of a Compact K\"ahler Manifold and the Complex Monge-Amp\`ere Equation, I*},
Communications on Pure and Applied Mathenatics, Vol. XXXXI, 339-411, John Wiley \& Sons. Inc.
\end{thebibliography}

\end{document}
